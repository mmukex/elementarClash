\section{Anforderungen (Max Meier)}
\label{chap:anforderungen}

% === ZIELUMFANG: ca. 2 Seiten ===

\subsection{Funktionale Anforderungen}
\label{sec:funktionale-anforderungen}

Die funktionalen Anforderungen orientieren sich an der Aufgabenstellung und wurden für ElementarClash wie folgt konkretisiert.

\begin{table}[H]
\centering
\begin{tabular}{l l p{9cm}}
\toprule
\textbf{ID} & \textbf{Anforderung} & \textbf{Beschreibung} \\
\midrule
F1 & Spielfeld & 2D-Raster (10$\times$10) mit unterschiedlichen Geländearten. Umgesetzt als 5 Geländetypen (Lava, Eis, Wald, Wüste, Stein) mit fester Verteilung: 30\,\% Wüste, 20\,\% Wald, 20\,\% Stein, 15\,\% Lava, 15\,\% Eis. \\
F2 & Einheiten & Verschiedene Einheitentypen, die sich bewegen und handeln können. Umgesetzt als 4 Fraktionen (Feuer, Wasser, Erde, Luft) mit je 3 Einheitentypen und individuellen Werten für LP (Lebenspunkte), Angriff, Verteidigung, Bewegung und Reichweite. \\
F3 & Rundenlogik & Spieler führen abwechselnd Züge aus; nach jeder Runde werden Ereignisse ausgewertet. Umgesetzt als Phasensystem (Setup, Spielerzug, Event, Spielende) mit 2 Aktionen pro Einheit und Runde. \\
F4 & Aktionen & Mindestens drei Aktionstypen. Umgesetzt als Bewegen (geländeabhängige Kosten), Angreifen (Nah-/Fernkampf) und Undo/Redo (Rücknahme aller Aktionen innerhalb einer Runde). \\
F5 & Siegbedingung & Das Spiel endet, wenn ein Spieler alle gegnerischen Einheiten besiegt hat. \\
F6 & Oberfläche & Einfache textuelle Darstellung des Spielfelds und der Spielzüge. Umgesetzt als ASCII-Konsolenoberfläche mit Eingabeaufforderungen. \\
\bottomrule
\end{tabular}
\caption{Funktionale Anforderungen (konkretisiert für ElementarClash)}
\label{tab:funktionale-anforderungen}
\end{table}

\subsection{Nicht-funktionale Anforderungen}
\label{sec:nicht-funktionale-anforderungen}

Die Aufgabenstellung fordert die Verwendung von Java, C\# oder TypeScript, mindestens 8 sinnvoll integrierte GoF-Patterns, Modularität und Erweiterbarkeit sowie UML-Dokumentation.

\begin{table}[H]
\centering
\begin{tabular}{l p{10cm}}
\toprule
\textbf{Anforderung} & \textbf{Umsetzung} \\
\midrule
Programmiersprache & Java 21 \\
Build-System & Gradle 8.14 mit Wrapper \\
Design Patterns & Gefordert: mindestens 8 GoF-Patterns aus den Kategorien Erzeugung, Struktur und Verhalten. Umgesetzt: 10 Patterns (siehe Kapitel~\ref{chap:architektur}, Tabelle~\ref{tab:pattern-uebersicht}). \\
Modularität & Erweiterbar ohne Änderungen am Core-Code (z.\,B. neue Fraktionen, Einheiten, Geländearten) durch konsequenten Einsatz von Interfaces und Polymorphie. \\
Dokumentation & PlantUML-Diagramme für alle Patterns, schriftliche Ausarbeitung in \LaTeX. \\
Tests & JUnit-5-Tests für alle 10 Design Patterns. \\
\bottomrule
\end{tabular}
\caption{Nicht-funktionale Anforderungen}
\label{tab:nicht-funktionale-anforderungen}
\end{table}

\subsection{Spielkonzept}
\label{sec:spielkonzept}

ElementarClash ist ein rundenbasiertes Strategiespiel, in dem zwei Spieler jeweils eine elementare Fraktion steuern.
Jede Fraktion verkörpert einen eigenen Spielstil: Feuer (aggressiv), Wasser (defensiv), Erde (kontrollierend) und Luft (mobil).

\subsubsection{Fraktionen und Einheiten}
\label{subsec:fraktionen-einheiten}

Jede Fraktion verfügt über drei Einheitentypen mit individuellen Werten.
Zusätzlich besitzt jede Fraktion einen passiven Bonus auf bestimmtem Gelände sowie einen Fraktionsvorteil (+25\,\% Schaden) und einen Fraktionsnachteil ($-25$\,\% Schaden) gegenüber je einer anderen Fraktion.

\begin{table}[H]
\centering
\begin{tabular}{l l r r r r r}
\toprule
\textbf{Fraktion} & \textbf{Einheit} & \textbf{LP} & \textbf{ATK} & \textbf{DEF} & \textbf{MOV} & \textbf{RNG} \\
\midrule
\multirow{3}{*}{Feuer}
 & Inferno-Krieger      & 100 & 15 &  5 & 3 & 1 \\
 & Flammen-Bogenschütze &  70 & 12 &  3 & 4 & 3 \\
 & Phönix               &  80 & 10 &  4 & 5 & 1 \\
\midrule
\multirow{3}{*}{Wasser}
 & Gezeiten-Wächter & 120 & 10 &  8 & 2 & 1 \\
 & Frost-Magier     &  60 & 13 &  4 & 3 & 4 \\
 & Wellen-Reiter    &  90 & 11 &  6 & 4 & 1 \\
\midrule
\multirow{3}{*}{Erde}
 & Stein-Golem      & 150 &  8 & 10 & 2 & 1 \\
 & Terra-Schamane   &  75 & 11 &  5 & 3 & 2 \\
 & Erdbeben-Titan   & 130 & 14 &  7 & 2 & 1 \\
\midrule
\multirow{3}{*}{Luft}
 & Wind-Tänzer      &  70 & 12 &  3 & 6 & 1 \\
 & Sturm-Rufer      &  65 & 14 &  2 & 4 & 3 \\
 & Himmels-Wächter  &  85 & 10 &  5 & 5 & 2 \\
\bottomrule
\end{tabular}
\caption{Einheitenübersicht aller vier Fraktionen}
\label{tab:einheiten}
\end{table}

\begin{table}[H]
\centering
\begin{tabular}{l r r r r}
\toprule
\textbf{Angreifer $\backslash$ Verteidiger} & \textbf{Feuer} & \textbf{Wasser} & \textbf{Erde} & \textbf{Luft} \\
\midrule
Feuer  &     0 & $-25$\,\% & $+25$\,\% &     0 \\
Wasser & $+25$\,\% &     0 & $-25$\,\% &     0 \\
Erde   &     0 & $+25$\,\% &     0 & $-25$\,\% \\
Luft   & $-25$\,\% &     0 & $+25$\,\% &     0 \\
\bottomrule
\end{tabular}
\caption{Fraktions-Korrelationsmatrix: Schadensmodifikator des Angreifers gegenüber dem Verteidiger}
\label{tab:fraktions-korrelationsmatrix}
\end{table}

\subsubsection{Geländearten}
\label{subsec:gelaendearten}

Das 10$\times$10-Spielfeld wird aus fünf Geländearten zusammengesetzt, die Bewegungskosten, Verteidigungsboni und fraktionsspezifische Effekte besitzen.

\begin{table}[H]
\centering
\begin{tabular}{l l l p{5.5cm}}
\toprule
\textbf{Gelände} & \textbf{Bew.-Kosten} & \textbf{DEF-Bonus} & \textbf{Fraktions-Effekte} \\
\midrule
Lava  & Normal: 2, Feuer: 1, Wasser: 3 & 0 & Feuer: +2 ATK; Wasser: $-5$ LP/Runde \\
Eis   & Normal: 3, Wasser: 1, Feuer: 2 & +1 & Wasser: +3 DEF, +5 LP/Runde; Feuer: +1 DEF, schmilzt zu Wüste \\
Wald  & Normal: 2 & +2 & Blockiert Fernkampf-Sichtlinie \\
Wüste & Normal: 1 & 0 & Neutral, keine Boni \\
Stein & Normal: 3, Erde: 2 & 0 & Erde: +2 DEF \\
\bottomrule
\end{tabular}
\caption{Geländearten mit Bewegungskosten und Effekten}
\label{tab:gelaende}
\end{table}

\subsubsection{Kern-Spielmechaniken}
\label{subsec:kern-spielmechaniken}

Jede Einheit kann pro Runde maximal zwei Aktionen ausführen: einmal Bewegen und einmal Angreifen.
Bewegungen kosten je nach Gelände und Fraktion unterschiedlich viele Bewegungspunkte; fliegende Einheiten (Luft-Fraktion sowie Phönix) ignorieren Geländekosten.
Angriffe unterscheiden sich in Nahkampf (Reichweite~1) und Fernkampf (Reichweite~2--4), wobei Wald-Gelände die Sichtlinie für Fernkämpfer blockieren kann.
Geländeeffekte werden per Visitor Pattern fraktionsspezifisch berechnet, Fraktionsboni und -mali per Chain of Responsibility in die Schadensberechnung integriert.
Mit steigender Rundenzahl können zufällige (De-)Buffs auf Einheiten angewendet werden, die über das Decorator Pattern gestapelt werden.
Alle Aktionen einer Runde sind per Undo/Redo rücknehmbar (Command Pattern).
