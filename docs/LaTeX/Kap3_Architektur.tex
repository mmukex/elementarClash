\chapter{Architektur}
\label{chap:architektur}

% === ZIELUMFANG: ca. 10 Seiten (Hauptteil) ===
% Pro Pattern: ca. 0.5–1 Seite + UML-Diagramm

% Einleitung: Übersichtstabelle aller 10 Patterns (wie in README)

\section{Erzeugungsmuster}
\label{sec:erzeugungsmuster}

\subsection{Factory Method (Christian Stiens)}
\label{subsec:factory-method}
% Anwendung: Fraktionsspezifische Einheiten-Erzeugung
% Begründung: 4 Fraktionen × 3 Einheiten, erweiterbar ohne Core-Änderungen
% UML-Diagramm + Code-Beispiel
% Verantwortlich: @crstmkt
Jede der vier Fraktion hat einen Basisstamm von drei Einheitentypen.
Jeder Einheitentyp hat einen Bassissatz an Stats (Lebenspunkte, Angriffspunkte, Verteidigungspunkte etc.).
Um eventuelle zukünftige Erweiterungen ohne Änderungen an der Core Logik zu ermöglichen und um nicht für jeden Einheitentypen den speziellen Kontruktor aufrufen zu müssen wir Factory Method verwedent.

\begin{figure}[H]
	\centering
	\includegraphics[width=\textwidth]{images/Factory_Method_Pattern___ElementarClash}
	\caption{Factory Method Pattern für die Einheitenerzeugung}
	\label{fig:factory-method}
\end{figure}

\subsection{Builder (Max Meier)}
\label{subsec:builder}
% Anwendung: Spielfeld-Erstellung (100 Zellen, Geländeverteilung, Startpositionen)
% Begründung: Komplexes Objekt schrittweise konfigurieren
% UML-Diagramm + Code-Beispiel
% Verantwortlich: @mmukex

\section{Strukturmuster}
\label{sec:strukturmuster}

\subsection{Composite (Max Meier)}
\label{subsec:composite}
% Anwendung: Battlefield → Regionen → Zellen
% Begründung: Einheitliche Operationen auf Teil und Ganzem
% UML-Diagramm
% Verantwortlich: @mmukex

\subsection{Decorator (Christian Stiens)}
\label{subsec:decorator}
% Anwendung: Temporäre Buffs/Debuffs stapeln
% Begründung: Dynamische Erweiterung zur Laufzeit
% UML-Diagramm
% Verantwortlich: @crstmkt
Im Spielverlauf wir mit steigender Rundenzahl die Wahrscheinlichkeit erhöhte, dass während der Eventphase eine zufällig ausgewählte Einheit der aktiven Fraktion einen (von sechs fest definierten) zufällig ausgewählten (De-)Buff erhält.
Dieser (De-)Buff wird per Decorator Pattern an die Unit angeklebt.
Im UI werden diese (De-)Buffs in geschweiften Klammern angezeigt.

\begin{figure}[H]
	\centering
	\includegraphics[width=\textwidth]{images/Decorator_Pattern___ElementarClash-Decorator_Pattern___ElementarClash_Unit_Buffs_Debuffs__GoF__4__Dynamically_Attaching_Responsibilities}
	\caption{Decorator Pattern für temporäre Buffs und Debuffs}
	\label{fig:decorator-pattern}
\end{figure}

\section{Verhaltensmuster}
\label{sec:verhaltensmuster}

\subsection{Strategy (Max Meier)}
\label{subsec:strategy}
% Anwendung: Bewegung (Boden/Fliegend), Angriff (Nahkampf/Fernkampf)
% Begründung: Austauschbare Algorithmen pro Fraktion
% UML-Diagramm
% Verantwortlich: @mmukex

\subsection{State (Christian Stiens)}
\label{subsec:state}
% Anwendung: Spielphasen (Setup, Event, InProgress, GameOver)
% Begründung: Zustandsabhängiges Verhalten
% UML-Diagramm
% Verantwortlich: @crstmkt
Spielphasen (SetupPhase, PlayerTurnPhase, EventPhase, GameOverPhase) und UnitStates (Attacking, Moving, Stunned, Idle, Dead) werden über das State Pattern implementiert.

\begin{figure}[H]
	\centering
	\includegraphics[width=\textwidth]{images/State_Pattern___ElementarClash-State_Pattern___ElementarClash_Game_Phases__GoF__6__Object_Behavior_Changes_Based_on_Internal_State}
	\caption{State Pattern für Spielphasen und Einheitenzustände}
	\label{fig:state-pattern}
\end{figure}

\subsection{Observer (Christian Stiens)}
\label{subsec:observer}
% Anwendung: Event-System (UI-Updates entkoppeln)
% Begründung: Lose Kopplung zwischen Spiellogik und Darstellung
% UML-Diagramm
% Verantwortlich: @crstmkt
Entkopplung von UI-Updates und dynamische Ereignissen.

\begin{figure}[H]
	\centering
	\includegraphics[width=\textwidth]{images/Observer_Pattern___ElementarClash-Observer_Pattern___ElementarClash_Event_System__GoF__7__One_to_Many_Dependency_for_Event_Notification}
	\caption{Observer Pattern für UI-Updates und Event-System}
	\label{fig:observer-pattern}
\end{figure}

\subsection{Command (Max Meier)}
\label{subsec:command}
% Anwendung: MoveCommand, AttackCommand + Undo/Redo
% Begründung: Aktionen als Objekte, CommandHistory für Rollback
% UML-Diagramm + Code-Beispiel
% Verantwortlich: @mmukex

\subsection{Chain of Responsibility (Christian Stiens)}
\label{subsec:chain}
% Anwendung: Schadensberechnung-Pipeline
% Begründung: Modulare Handler-Kette für Modifikatoren
% UML-Diagramm
% Verantwortlich: @crstmkt
Basisschaden, Fraktionseffekte, Terraineffekte, (temporäre) Effekte auf einer Einheit addieren sich zu einem Gesamtschaden, den die Einheit erleidet.


\begin{figure}[H]
	\centering
	\includegraphics[width=\textwidth]{images/Chain_of_Responsibility_Pattern___ElementarClash-Chain_of_Responsibility_Pattern___ElementarClash_Damage_Pipeline__GoF__9__Decoupled_Request_Processing_Through_Handler_Chain}
	\caption{Chain of Responsibility Pattern für modulare Schadensberechnung}
	\label{fig:chain-of-responsibility}
\end{figure}

\subsection{Visitor (Max Meier)}
\label{subsec:visitor}
% Anwendung: Gelände-Effekte auf Einheiten (5 Gelände × 4 Fraktionen)
% Begründung: Double Dispatch, vermeidet if-Verschachtelung
% UML-Diagramm
% Verantwortlich: @mmukex

\section{Programmentwurf}
\label{sec:programmentwurf}
% Gesamtübersicht: Paketstruktur-Diagramm
% Optional: Sequenzdiagramm Angriff
