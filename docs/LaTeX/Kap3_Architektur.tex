\chapter{Architektur}
\label{chap:architektur}

% === ZIELUMFANG: ca. 10 Seiten (Hauptteil) ===
% Pro Pattern: ca. 0.5–1 Seite + UML-Diagramm

% Einleitung: Übersichtstabelle aller 10 Patterns (wie in README)

\section{Erzeugungsmuster}
\label{sec:erzeugungsmuster}

\subsection{Factory Method}
\label{subsec:factory-method}
% Anwendung: Fraktionsspezifische Einheiten-Erzeugung
% Begründung: 4 Fraktionen × 3 Einheiten, erweiterbar ohne Core-Änderungen
% UML-Diagramm + Code-Beispiel
% Verantwortlich: @crstmkt

\subsection{Builder}
\label{subsec:builder}
% Anwendung: Spielfeld-Erstellung (100 Zellen, Geländeverteilung, Startpositionen)
% Begründung: Komplexes Objekt schrittweise konfigurieren
% UML-Diagramm + Code-Beispiel
% Verantwortlich: @mmukex

\section{Strukturmuster}
\label{sec:strukturmuster}

\subsection{Composite}
\label{subsec:composite}
% Anwendung: Battlefield → Regionen → Zellen
% Begründung: Einheitliche Operationen auf Teil und Ganzem
% UML-Diagramm
% Verantwortlich: @mmukex

\subsection{Decorator}
\label{subsec:decorator}
% Anwendung: Temporäre Buffs/Debuffs stapeln
% Begründung: Dynamische Erweiterung zur Laufzeit
% UML-Diagramm
% Verantwortlich: @crstmkt

\section{Verhaltensmuster}
\label{sec:verhaltensmuster}

\subsection{Strategy}
\label{subsec:strategy}
% Anwendung: Bewegung (Boden/Fliegend), Angriff (Nahkampf/Fernkampf)
% Begründung: Austauschbare Algorithmen pro Fraktion
% UML-Diagramm
% Verantwortlich: @mmukex

\subsection{State}
\label{subsec:state}
% Anwendung: Spielphasen (Setup, InProgress, GameOver)
% Begründung: Zustandsabhängiges Verhalten
% UML-Diagramm
% Verantwortlich: @crstmkt

\subsection{Observer}
\label{subsec:observer}
% Anwendung: Event-System (UI-Updates entkoppeln)
% Begründung: Lose Kopplung zwischen Spiellogik und Darstellung
% UML-Diagramm
% Verantwortlich: @crstmkt

\subsection{Command}
\label{subsec:command}
% Anwendung: MoveCommand, AttackCommand + Undo/Redo
% Begründung: Aktionen als Objekte, CommandHistory für Rollback
% UML-Diagramm + Code-Beispiel
% Verantwortlich: @mmukex

\subsection{Chain of Responsibility}
\label{subsec:chain}
% Anwendung: Schadensberechnung-Pipeline
% Begründung: Modulare Handler-Kette für Modifikatoren
% UML-Diagramm
% Verantwortlich: @crstmkt

\subsection{Visitor}
\label{subsec:visitor}
% Anwendung: Gelände-Effekte auf Einheiten (5 Gelände × 4 Fraktionen)
% Begründung: Double Dispatch, vermeidet if-Verschachtelung
% UML-Diagramm
% Verantwortlich: @mmukex

\section{Programmentwurf}
\label{sec:programmentwurf}
% Gesamtübersicht: Paketstruktur-Diagramm
% Optional: Sequenzdiagramm Angriff
