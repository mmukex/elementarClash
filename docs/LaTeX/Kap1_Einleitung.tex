\chapter{Einleitung (Christian Stiens)}
\label{chap:einleitung}

% === ZIELUMFANG: ca. 1 Seite ===

% Inhalt:
% - Motivation: Praktische Anwendung von GoF-Patterns in einem Spielprojekt
% - Spielidee ElementarClash: 4 Fraktionen (Feuer, Wasser, Erde, Luft), 
%   10×10 Raster, 5 Geländearten, elementare Synergien
% - Kurzer Aufbau der Arbeit (1 Absatz)

Diese schriftliche Ausarbeitung wurde begleitend zum Projekt \textit{ElementarClash} von Max Meier und Christian Stiens für das Fach \textit{Design Pattern in der OOP} erstellt.
Ziel der Entwicklung war die praktische Anwendung von Design Pattern nach GoF in einem lauffähigen Programm.
\textit{ElementarClash} ist ein rundenbasiertes Strategiespiel.
Jeder Spieler steuert eine von bis zu vier Fraktionen (Feuer, Wasser, Erde, Luft) auf einem 10x10 Spielfeld mit 5 unterschiedlichen Terrains (Geländearten).
Diese Terrains wiederum können sich positiv oder negativ auf die Fähigkeiten der Spielfiguren auswirken.
Ziel des Spiels ist es alle gegnerischen Einheiten zu besiegen.
Zur Umsetzung der verschiedenen Spielmechaniken (Siehe Kapitel 4.) wurden 10 Design Pattern nach GoF gewählt, welche in Kapitel 3 mit ihren jeweilig zugedachten Funktionen vorgestellt werden.

