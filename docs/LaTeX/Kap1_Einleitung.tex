\section{Einleitung (Christian Stiens)}
\label{chap:einleitung}

% === ZIELUMFANG: ca. 1 Seite ===

% Inhalt:
% - Motivation: Praktische Anwendung von GoF-Patterns in einem Spielprojekt
% - Spielidee ElementarClash: 4 Fraktionen (Feuer, Wasser, Erde, Luft), 
%   10×10 Raster, 5 Geländearten, elementare Synergien
% - Kurzer Aufbau der Arbeit (1 Absatz)

Diese schriftliche Ausarbeitung wurde begleitend zum Projekt \textit{ElementarClash} von Max Meier und Christian Stiens für das Fach \textit{Design Pattern in der OOP} erstellt.
Ziel der Entwicklung war die praktische Anwendung von Design Pattern nach GoF in einem lauffähigen Programm.
ElementarClash ist ein rundenbasiertes Strategiespiel, in dem jeder Spieler eine von bis zu vier Fraktionen (Feuer, Wasser, Erde, Luft) auf einem 10×10-Spielfeld mit fünf unterschiedlichen Geländearten steuert.
Diese Geländearten wirken sich positiv oder negativ auf die Fähigkeiten der Einheiten aus. Ziel des Spiels ist es, alle gegnerischen Einheiten zu besiegen.
Die Wahl eines Strategiespiels als Projektrahmen ergab sich aus der Eignung des Genres für den Einsatz von Design Patterns: Spielmechaniken wie Einheitenerzeugung, Bewegungslogik, Schadensberechnung und Zustandsverwaltung lassen sich klar voneinander trennen und durch geeignete Patterns strukturieren.
Dabei wurden bewusst alle drei GoF-Kategorien berücksichtigt – Erzeugungsmuster für die Spielaufbaulogik, Strukturmuster für die Spielfeldhierarchie und Buff-Verwaltung sowie Verhaltensmuster für Aktionen, Phasen und Geländeeffekte.
Kapitel 2 konkretisiert die funktionalen und nicht-funktionalen Anforderungen an das Spiel.
Kapitel 3 stellt die zehn gewählten Design Patterns mit ihrem jeweiligen Anwendungsfall vor.
Kapitel 4 beschreibt die Implementierung der zentralen Spielkomponenten.
Kapitel 5 fasst die Ergebnisse zusammen und gibt einen Ausblick auf mögliche Erweiterungen.

