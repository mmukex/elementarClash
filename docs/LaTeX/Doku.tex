\documentclass[a4paper, numbers=withenddot, 11pt, twoside, openany]{scrbook}

% Vorlage für Ausarbeitungen
% Kompiliere mit: lualatex Doku.tex
% Version 2024-09-26

% Hier Daten eintragen:
\newcommand{\name}{Max Meier \\ Matrikelnummer: 30394407 \\ Christian Stiens \\ Matrikelnummer: 30394676}
\newcommand{\modulname}{Design Pattern in der OOP}
\newcommand{\thema}{ElementarClash}
\newcommand{\hochschule}{Fachhochschule Südwestfalen}
\newcommand{\datum}{\today}
% Ende der Daten

% Systemschriftarten mit fontspec (LuaLaTeX/XeLaTeX)
\usepackage{fontspec}

% Automatische Schriftauswahl mit Fallback
\IfFontExistsTF{Adwaita Sans}{
    % Linux/Gnome: Adwaita Sans
    \setsansfont{Adwaita Sans}
}{
    \IfFontExistsTF{Segoe UI}{
        % Windows: Segoe UI
        \setsansfont{Segoe UI}
    }{
        % Fallback: Latin Modern Sans
        \setsansfont{Latin Modern Sans}
    }
}

\renewcommand{\familydefault}{\sfdefault}
\usepackage[ngerman]{babel}
\usepackage{booktabs}
\usepackage{multirow}
\usepackage{microtype}
\usepackage{graphicx}
\usepackage{float}
\usepackage{tikz}
\usepackage{scrhack}
\setkomafont{disposition}{\normalfont\bfseries}

% Kopf- und Fußzeilen
\usepackage[automark]{scrlayer-scrpage}
\clearpairofpagestyles
\ihead{\leftmark}  % Innen: Kapitelnummer und -titel
\ofoot*{\pagemark}  % Außen: Seitenzahl
\pagestyle{scrheadings}
\setlength{\footheight}{10pt}  % Reduzierte Höhe der Fußzeile

% Kopfzeile auch auf Kapitelanfangsseiten
\renewcommand*{\chapterpagestyle}{scrheadings}

% Weniger Abstand über Kapiteltitel
\renewcommand*{\chapterheadstartvskip}{\vspace*{1cm}}
\usepackage{setspace}\setstretch{1.1}
\parindent0pt\parskip6pt
\clubpenalty=10000
\widowpenalty=10000
\displaywidowpenalty=10000

\usepackage{listings}
\renewcommand{\lstlistingname}{Listing}
\lstset{basicstyle=\small\ttfamily, breaklines=true, keepspaces=true, columns=fixed}

% Hyperref für klickbare Links (optional, aber empfohlen)
\usepackage{hyperref}
\hypersetup{
    colorlinks=true,
    linkcolor=black,
    urlcolor=blue,
    pdftitle={\thema},
    pdfauthor={Max Meier, Christian Stiens}
}

\title{\thema}
\author{\name\\ \hochschule \\[5mm] Schriftliche Ausarbeitung im Modul\\ „\modulname"}

\begin{document}

\maketitle
\tableofcontents

% ============================================
% Kapitel einbinden
% ============================================
\section*{Erklärung}
\markboth{Erklärung}{Erklärung}

Ich erkläre hiermit, dass ich die vorliegende Arbeit selbstständig verfasst und dabei keine anderen als die angegebenen Hilfsmittel benutzt habe. Sämtliche Stellen der Arbeit, die im Wortlaut oder dem Sinn nach Werken anderer Autoren entnommen sind, habe ich als solche kenntlich gemacht. Die Arbeit wurde bisher weder gesamt noch in Teilen einer anderen Prüfungsbehörde vorgelegt und auch noch nicht veröffentlicht.

\bigskip
\noindent
\datum

\vspace{25mm}

\noindent
\name

\chapter{Einleitung (Christian Stiens)}
\label{chap:einleitung}

% === ZIELUMFANG: ca. 1 Seite ===

% Inhalt:
% - Motivation: Praktische Anwendung von GoF-Patterns in einem Spielprojekt
% - Spielidee ElementarClash: 4 Fraktionen (Feuer, Wasser, Erde, Luft), 
%   10×10 Raster, 5 Geländearten, elementare Synergien
% - Kurzer Aufbau der Arbeit (1 Absatz)

Diese schriftliche Ausarbeitung wurde begleitend zum Projekt \textit{ElementarClash} von Max Meier und Christian Stiens für das Fach \textit{Design Pattern in der OOP} erstellt.
Ziel der Entwicklung war die praktische Anwendung von Design Pattern nach GoF in einem lauffähigen Programm.
\textit{ElementarClash} ist ein rundenbasiertes Strategiespiel.
Jeder Spieler steuert eine von bis zu vier Fraktionen (Feuer, Wasser, Erde, Luft) auf einem 10x10 Spielfeld mit 5 unterschiedlichen Terrains (Geländearten).
Diese Terrains wiederum können sich positiv oder negativ auf die Fähigkeiten der Spielfiguren auswirken.
Ziel des Spiels ist es alle gegnerischen Einheiten zu besiegen.
Zur Umsetzung der verschiedenen Spielmechaniken (Siehe Kapitel 4.) wurden 10 Design Pattern nach GoF gewählt, welche in Kapitel 3 mit ihren jeweilig zugedachten Funktionen vorgestellt werden.


\section{Anforderungen (Max Meier)}
\label{chap:anforderungen}

% === ZIELUMFANG: ca. 2 Seiten ===

\subsection{Funktionale Anforderungen}
\label{sec:funktionale-anforderungen}

Die funktionalen Anforderungen orientieren sich an der Aufgabenstellung und wurden für ElementarClash wie folgt konkretisiert.

\begin{table}[H]
\centering
\begin{tabular}{l l p{9cm}}
\toprule
\textbf{ID} & \textbf{Anforderung} & \textbf{Beschreibung} \\
\midrule
F1 & Spielfeld & 2D-Raster (10$\times$10) mit unterschiedlichen Geländearten. Umgesetzt als 5 Geländetypen (Lava, Eis, Wald, Wüste, Stein) mit fester Verteilung: 30\,\% Wüste, 20\,\% Wald, 20\,\% Stein, 15\,\% Lava, 15\,\% Eis. \\
F2 & Einheiten & Verschiedene Einheitentypen, die sich bewegen und handeln können. Umgesetzt als 4 Fraktionen (Feuer, Wasser, Erde, Luft) mit je 3 Einheitentypen und individuellen Werten für LP (Lebenspunkte), Angriff, Verteidigung, Bewegung und Reichweite. \\
F3 & Rundenlogik & Spieler führen abwechselnd Züge aus; nach jeder Runde werden Ereignisse ausgewertet. Umgesetzt als Phasensystem (Setup, Spielerzug, Event, Spielende) mit 2 Aktionen pro Einheit und Runde. \\
F4 & Aktionen & Mindestens drei Aktionstypen. Umgesetzt als Bewegen (geländeabhängige Kosten), Angreifen (Nah-/Fernkampf) und Undo/Redo (Rücknahme aller Aktionen innerhalb einer Runde). \\
F5 & Siegbedingung & Das Spiel endet, wenn ein Spieler alle gegnerischen Einheiten besiegt hat. \\
F6 & Oberfläche & Einfache textuelle Darstellung des Spielfelds und der Spielzüge. Umgesetzt als ASCII-Konsolenoberfläche mit Eingabeaufforderungen. \\
\bottomrule
\end{tabular}
\caption{Funktionale Anforderungen (konkretisiert für ElementarClash)}
\label{tab:funktionale-anforderungen}
\end{table}

\subsection{Nicht-funktionale Anforderungen}
\label{sec:nicht-funktionale-anforderungen}

Die Aufgabenstellung fordert die Verwendung von Java, C\# oder TypeScript, mindestens 8 sinnvoll integrierte GoF-Patterns, Modularität und Erweiterbarkeit sowie UML-Dokumentation.

\begin{table}[H]
\centering
\begin{tabular}{l p{10cm}}
\toprule
\textbf{Anforderung} & \textbf{Umsetzung} \\
\midrule
Programmiersprache & Java 21 \\
Build-System & Gradle 8.14 mit Wrapper \\
Design Patterns & Gefordert: mindestens 8 GoF-Patterns aus den Kategorien Erzeugung, Struktur und Verhalten. Umgesetzt: 10 Patterns (siehe Kapitel~\ref{chap:architektur}, Tabelle~\ref{tab:pattern-uebersicht}). \\
Modularität & Erweiterbar ohne Änderungen am Core-Code (z.\,B. neue Fraktionen, Einheiten, Geländearten) durch konsequenten Einsatz von Interfaces und Polymorphie. \\
Dokumentation & PlantUML-Diagramme für alle Patterns, schriftliche Ausarbeitung in \LaTeX. \\
Tests & JUnit-5-Tests für alle 10 Design Patterns. \\
\bottomrule
\end{tabular}
\caption{Nicht-funktionale Anforderungen}
\label{tab:nicht-funktionale-anforderungen}
\end{table}

\subsection{Spielkonzept}
\label{sec:spielkonzept}

ElementarClash ist ein rundenbasiertes Strategiespiel, in dem zwei Spieler jeweils eine elementare Fraktion steuern.
Jede Fraktion verkörpert einen eigenen Spielstil: Feuer (aggressiv), Wasser (defensiv), Erde (kontrollierend) und Luft (mobil).

\subsubsection{Fraktionen und Einheiten}
\label{subsec:fraktionen-einheiten}

Jede Fraktion verfügt über drei Einheitentypen mit individuellen Werten.
Zusätzlich besitzt jede Fraktion einen passiven Bonus auf bestimmtem Gelände sowie einen Fraktionsvorteil (+25\,\% Schaden) und einen Fraktionsnachteil ($-25$\,\% Schaden) gegenüber je einer anderen Fraktion.

\begin{table}[H]
\centering
\begin{tabular}{l l r r r r r}
\toprule
\textbf{Fraktion} & \textbf{Einheit} & \textbf{LP} & \textbf{ATK} & \textbf{DEF} & \textbf{MOV} & \textbf{RNG} \\
\midrule
\multirow{3}{*}{Feuer}
 & Inferno-Krieger      & 100 & 15 &  5 & 3 & 1 \\
 & Flammen-Bogenschütze &  70 & 12 &  3 & 4 & 3 \\
 & Phönix               &  80 & 10 &  4 & 5 & 1 \\
\midrule
\multirow{3}{*}{Wasser}
 & Gezeiten-Wächter & 120 & 10 &  8 & 2 & 1 \\
 & Frost-Magier     &  60 & 13 &  4 & 3 & 4 \\
 & Wellen-Reiter    &  90 & 11 &  6 & 4 & 1 \\
\midrule
\multirow{3}{*}{Erde}
 & Stein-Golem      & 150 &  8 & 10 & 2 & 1 \\
 & Terra-Schamane   &  75 & 11 &  5 & 3 & 2 \\
 & Erdbeben-Titan   & 130 & 14 &  7 & 2 & 1 \\
\midrule
\multirow{3}{*}{Luft}
 & Wind-Tänzer      &  70 & 12 &  3 & 6 & 1 \\
 & Sturm-Rufer      &  65 & 14 &  2 & 4 & 3 \\
 & Himmels-Wächter  &  85 & 10 &  5 & 5 & 2 \\
\bottomrule
\end{tabular}
\caption{Einheitenübersicht aller vier Fraktionen}
\label{tab:einheiten}
\end{table}

\begin{table}[H]
\centering
\begin{tabular}{l r r r r}
\toprule
\textbf{Angreifer $\backslash$ Verteidiger} & \textbf{Feuer} & \textbf{Wasser} & \textbf{Erde} & \textbf{Luft} \\
\midrule
Feuer  &     0 & $-25$\,\% & $+25$\,\% &     0 \\
Wasser & $+25$\,\% &     0 & $-25$\,\% &     0 \\
Erde   &     0 & $+25$\,\% &     0 & $-25$\,\% \\
Luft   & $-25$\,\% &     0 & $+25$\,\% &     0 \\
\bottomrule
\end{tabular}
\caption{Fraktions-Korrelationsmatrix: Schadensmodifikator des Angreifers gegenüber dem Verteidiger}
\label{tab:fraktions-korrelationsmatrix}
\end{table}

\subsubsection{Geländearten}
\label{subsec:gelaendearten}

Das 10$\times$10-Spielfeld wird aus fünf Geländearten zusammengesetzt, die Bewegungskosten, Verteidigungsboni und fraktionsspezifische Effekte besitzen.

\begin{table}[H]
\centering
\begin{tabular}{l l l p{5.5cm}}
\toprule
\textbf{Gelände} & \textbf{Bew.-Kosten} & \textbf{DEF-Bonus} & \textbf{Fraktions-Effekte} \\
\midrule
Lava  & Normal: 2, Feuer: 1, Wasser: 3 & 0 & Feuer: +2 ATK; Wasser: $-5$ LP/Runde \\
Eis   & Normal: 3, Wasser: 1, Feuer: 2 & +1 & Wasser: +3 DEF, +5 LP/Runde; Feuer: +1 DEF, schmilzt zu Wüste \\
Wald  & Normal: 2 & +2 & Blockiert Fernkampf-Sichtlinie \\
Wüste & Normal: 1 & 0 & Neutral, keine Boni \\
Stein & Normal: 3, Erde: 2 & 0 & Erde: +2 DEF \\
\bottomrule
\end{tabular}
\caption{Geländearten mit Bewegungskosten und Effekten}
\label{tab:gelaende}
\end{table}

\subsubsection{Kern-Spielmechaniken}
\label{subsec:kern-spielmechaniken}

Jede Einheit kann pro Runde maximal zwei Aktionen ausführen: einmal Bewegen und einmal Angreifen.
Bewegungen kosten je nach Gelände und Fraktion unterschiedlich viele Bewegungspunkte; fliegende Einheiten (Luft-Fraktion sowie Phönix) ignorieren Geländekosten.
Angriffe unterscheiden sich in Nahkampf (Reichweite~1) und Fernkampf (Reichweite~2--4), wobei Wald-Gelände die Sichtlinie für Fernkämpfer blockieren kann.
Geländeeffekte werden per Visitor Pattern fraktionsspezifisch berechnet, Fraktionsboni und -mali per Chain of Responsibility in die Schadensberechnung integriert.
Mit steigender Rundenzahl können zufällige (De-)Buffs auf Einheiten angewendet werden, die über das Decorator Pattern gestapelt werden.
Alle Aktionen einer Runde sind per Undo/Redo rücknehmbar (Command Pattern).

\section{Architektur (Christian Stiens \& Max Meier)}
\label{chap:architektur}

% === ZIELUMFANG: ca. 10 Seiten (Hauptteil) ===
% Pro Pattern: ca. 0.5–1 Seite + UML-Diagramm

ElementarClash setzt 10 GoF-Patterns aus allen drei Kategorien ein.
Tabelle~\ref{tab:pattern-uebersicht} gibt einen Überblick über die gewählten Patterns, ihren konkreten Anwendungsfall im Spiel und die Begründung für die Wahl.
Die anschließenden Abschnitte beschreiben jedes Pattern im Detail.
In den zugehörigen UML-Diagrammen wurde zugunsten der Übersichtlichkeit auf die Darstellung von Packages verzichtet.
Implementieren mehrere konkrete Klassen dasselbe Interface, wird exemplarisch nur eine dargestellt.

\begin{table}[H]
\centering
\small
\begin{tabular}{r l l p{3.5cm} p{4.5cm}}
\toprule
\textbf{\#} & \textbf{Pattern} & \textbf{Kat.} & \textbf{Anwendungsfall} & \textbf{Begründung} \\
\midrule
1 & Factory Method & Erz. & Fraktionsspezifische Einheiten-Erzeugung & 4 Fraktionen $\times$ 3 Typen; erweiterbar ohne Core-Änderungen \\
2 & Builder & Erz. & Schrittweise Spielfeld-Erstellung & 100 Zellen, Geländeverteilung, Spawn-Positionen -- zu komplex für einen Konstruktor \\
3 & Composite & Str. & Battlefield $\to$ Region $\to$ Cell & Einheitliche Operationen auf Einzelzellen und ganzen Regionen \\
4 & Decorator & Str. & Temporäre (De-)Buffs & Stapelbare Laufzeit-Erweiterungen ohne Unterklassen \\
5 & Strategy & Verh. & Bewegung (Boden/Fliegend), Angriff (Nah-/Fernkampf) & Austauschbare Algorithmen pro Einheit \\
6 & State & Verh. & Spielphasen und Einheitenzustände & Zustandsabhängiges Verhalten ohne \texttt{if}-Kaskaden \\
7 & Observer & Verh. & Event-System, UI-Entkopplung & Lose Kopplung zwischen Spiellogik und Darstellung \\
8 & Command & Verh. & MoveCommand, AttackCommand + Undo/Redo & Aktionen als Objekte mit Rollback-Fähigkeit \\
9 & Chain of Resp. & Verh. & Schadensberechnungs-Pipeline & Modulare Handler-Kette für Modifikatoren \\
10 & Visitor & Verh. & Gelände-Effekte auf Einheiten & Double Dispatch für 5 Gelände $\times$ 4 Fraktionen \\
\bottomrule
\end{tabular}
\caption{Übersicht der eingesetzten Design Patterns}
\label{tab:pattern-uebersicht}
\end{table}

\subsection{Programmentwurf (Max Meier)}
\label{sec:programmentwurf}

Die Codebasis ist in fünf Hauptpakete gegliedert, die jeweils eine fachliche Verantwortung kapseln:

\begin{description}
	\item[\texttt{battlefield}]
	Spielfeld-Modell (Composite Pattern): \texttt{Cell}, \texttt{Region}, \texttt{Battlefield} sowie das Unter\-paket \texttt{terraineffect} mit dem Visitor Pattern (\texttt{TerrainVisitor}, fünf konkrete Visitor, \texttt{TerrainVisitorFactory}).

	\item[\texttt{units}]
	Einheiten-Modell: Basisklasse \texttt{Unit}, Fraktions-Enum und die Unter\-pakete \texttt{types} (12~konkrete Einheiten in je einem Fraktionspaket), \texttt{strategy} (Movement-/Attack\-Strategy), \texttt{state} (Unit\-States) sowie \texttt{bonus} mit \texttt{temporary} (Decorator Pattern für Buffs/Debuffs).

	\item[\texttt{game}]
	Spiellogik: \texttt{Game}, \texttt{GameBuilder} (Builder Pattern), \texttt{RoundManager} sowie die Unter\-pakete \texttt{phase} (State Pattern für Spielphasen), \texttt{command} (Command Pattern mit \texttt{CommandExecutor} und \texttt{CommandHistory}), \texttt{combat} mit \texttt{handler} (Chain of Responsibility) und \texttt{event} mit \texttt{observer} (Observer Pattern).

	\item[\texttt{ui}]
	ASCII-Konsolenoberfläche: Rendering des Spielfelds und Verarbeitung der Spielereingaben.

	\item[\texttt{util}]
	Hilfsklassen, u.\,a. \texttt{Position} für Koordinatenberechnungen.
\end{description}

Diese Paketstruktur spiegelt die Trennung von Datenmodell (\texttt{battlefield}, \texttt{units}), Spiellogik (\texttt{game}) und Darstellung (\texttt{ui}) wider.
Jedes der 10~Design Patterns ist in dem Paket verortet, das seine fachliche Domäne repräsentiert -- beispielsweise liegt das Strategy Pattern unter \texttt{units.strategy}, das Visitor Pattern unter \texttt{battlefield.terraineffect} und das Command Pattern unter \texttt{game.command}.

Der \texttt{GameBuilder} (Builder Pattern) erzeugt das \texttt{Battlefield} als Composite-Struktur aus \texttt{Cell}, \texttt{Region} und \texttt{Battlefield}.
Der \texttt{TerrainEffectHandler} in der Schadensberechnungs-Pipeline (Chain of Responsibility) dispatcht für Angreifer und Verteidiger je einen \texttt{TerrainVisitor} auf der aktuellen Zelle des Composite-Spielfelds.
\texttt{MoveCommand} und \texttt{AttackCommand} (Command Pattern) delegieren ihre Validierung an die jeweilige Strategy der Einheit.
Der \texttt{AttackCommand} löst die Handler-Kette des \texttt{DamageCalculator} aus, die Visitor, Decorator und Fraktionsmatrix integriert.
State Pattern und Observer Pattern rahmen den Ablauf ein: Spielphasen steuern den Kontrollfluss, Observer entkoppeln die UI von der Spiellogik.

\subsection{Erzeugungsmuster}
\label{sec:erzeugungsmuster}

\subsubsection{Factory Method (Christian Stiens)}
\label{subsec:factory-method}
% Anwendung: Fraktionsspezifische Einheiten-Erzeugung
% Begründung: 4 Fraktionen × 3 Einheiten, erweiterbar ohne Core-Änderungen
% UML-Diagramm + Code-Beispiel
% Verantwortlich: @crstmkt
ElementarClash umfasst 12 konkrete Einheitenklassen in vier Fraktionen mit je eigenen Stat-Werten (vgl. Kapitel 2.3.1).
Ohne ein Erzeugungsmuster müsste der Aufrufer jeden konkreten Konstruktor kennen und die passenden Stat-Werte selbst übergeben -- eine Kopplung, die bei einer neuen Fraktion Änderungen an mehreren Stellen im Code erfordern würde.

Das Factory Method Pattern löst dieses Problem durch eine zweistufige Hierarchie.
Die abstrakte Klasse \texttt{UnitFactory} definiert \texttt{createUnit(UnitType)} als finalen Template-Method-Einstiegspunkt: Sie validiert, ob der angeforderte \texttt{UnitType} zur Fraktion gehört, generiert eine eindeutige Einheiten-ID im Format \texttt{F1}, \texttt{W2} usw. (Anfangsbuchstabe der Fraktion + laufender Zähler) und delegiert die eigentliche Objekterzeugung an die abstrakte Factory Method \texttt{createUnitInternal(id, type)}.
Jede der vier konkreten Fabriken -- \texttt{FireUnitFactory}, \texttt{WaterUnitFactory}, \texttt{EarthUnitFactory} und \texttt{AirUnitFactory} -- implementiert \texttt{createUnitInternal()} als \texttt{switch}-Ausdruck über den \texttt{UnitType} und instanziiert die zugehörige Einheitenklasse mit den hart codierten Stat-Werten (vgl. \texttt{FireUnitFactory.java}).
Die konkreten Einheitenklassen sind package-private, sodass sie ausschließlich über ihre Fabrik instanziiert werden können -- eine durch die Paketstruktur erzwungene Kapselung. (vgl. \cite{entwurfsmuster})

Das Ergebnis: Das Hinzufügen einer neuen Fraktion erfordert lediglich eine neue Fabrik und die zugehörigen Einheitenklassen im gleichen Paket.
Die restliche Spiellogik -- \texttt{GameBuilder}, \texttt{Game}, \texttt{DamageCalculator} -- bleibt unverändert und arbeitet ausschließlich gegen die abstrakte \texttt{UnitFactory} und \texttt{Unit}.

\begin{figure}[H]
	\centering
	\includegraphics[width=\textwidth]{images/Factory_Method_Pattern___ElementarClash}
	\caption{Factory Method Pattern für die Einheitenerzeugung}
	\label{fig:factory-method}
\end{figure}

\subsubsection{Builder (Max Meier)}
\label{subsec:builder}
% Verantwortlich: @mmukex
Das 10$\times$10-Spielfeld umfasst 100 Zellen mit anteiliger Geländeverteilung (30\,\% Wüste, 20\,\% Wald, 20\,\% Stein, 15\,\% Lava, 15\,\% Eis), Startpositionen für zwei Fraktionen und eine beliebige Anzahl von Einheiten pro Fraktion.
Ein einzelner Konstruktoraufruf ist für diese Komplexität unpraktikabel.
Das Builder Pattern trennt die schrittweise Konfiguration von der eigentlichen Objekterzeugung (vgl. \cite{gof}).

Die Klasse \texttt{GameBuilder} stellt eine Fluent-API bereit: \texttt{withFactions()} registriert Fraktionen, \texttt{withCustomTerrain()} setzt optional eine benutzerdefinierte Geländeverteilung, \texttt{withRandomSeed()} einen Zufalls-Seed.
Einheiten werden per \texttt{addUnit()} einer Fraktion zugewiesen.
Die \texttt{build()}-Methode validiert die Konfiguration (mindestens 2 Fraktionen, jede mit Einheiten), erzeugt das \texttt{Battlefield}, verteilt die Geländetypen nach Prozentverteilung und platziert die Einheiten in zwei gegenüberliegenden Ecken-Spawn-Zonen à 3$\times$3~Felder (vgl. \texttt{GameBuilder.java Z.\,87ff}).
Unterschiedliche Spielkonfigurationen (z.\,B. reine Lava-Karte, symmetrische Wettkampfkarte) lassen sich so mit derselben Builder-Klasse erzeugen.

\begin{figure}[H]
	\centering
	\includegraphics[width=\textwidth]{images/Builder_Pattern___ElementarClash}
	\caption{Builder Pattern für die Spielerstellung}
	\label{fig:builder-pattern}
\end{figure}

\subsection{Strukturmuster}
\label{sec:strukturmuster}

\subsubsection{Composite (Max Meier)}
\label{subsec:composite}
% Verantwortlich: @mmukex
Das Spielfeld besteht aus 100 einzelnen Zellen, die in Zeilen und beliebigen Teilregionen organisiert sind.
Geländeeffekte und dynamische Ereignisse (z.\,B. Waldbrand, Erdbeben, Geysir) müssen einheitlich auf einzelne Zellen und ganze Regionen anwendbar sein.
Das Composite Pattern definiert dafür eine gemeinsame Schnittstelle für Einzel- und Gruppenobjekte (vgl. \cite{gof}).

Das Interface \texttt{BattlefieldComponent} deklariert \texttt{cells()}, \texttt{getCell(int)} und \texttt{applyEffect(Consumer<Cell>)}.
\texttt{Cell} (Leaf) gibt eine Einelementliste zurück, \texttt{Region} (Composite) delegiert an ihre enthaltenen Zellen.
\texttt{Battlefield} (Root-Composite) organisiert 10~Regionen à 10~Zellen und bietet \texttt{getRegion(x1,\,y1,\,x2,\,y2)} für beliebige Teilbereiche (vgl. \texttt{Battlefield.java Z.\,119ff}).
Events der \texttt{EventPhase} wenden ihre Effekte per \texttt{applyEffect()} auf eine einzelne Zelle, eine 3$\times$3-Region oder das gesamte Spielfeld an -- ohne duplizierte Logik.

\begin{figure}[H]
	\centering
	\includegraphics[width=\textwidth]{images/Composite_Pattern___ElementarClash}
	\caption{Composite Pattern für die Spielfeld-Hierarchie}
	\label{fig:composite-pattern}
\end{figure}

\subsubsection{Decorator (Christian Stiens)}
\label{subsec:decorator}
% Anwendung: Temporäre Buffs/Debuffs stapeln
% Begründung: Dynamische Erweiterung zur Laufzeit
% UML-Diagramm
% Verantwortlich: @crstmkt
Einheiten können zur Laufzeit eine beliebige Kombination aus Angriffs-, Verteidigungs- und Bewegungsboni erhalten, die aus unterschiedlichen Quellen stammen: dem Geländeeffekt des Feldes, dem Fraktions-Synergiebonus sowie zufälligen Ereignissen.
Eine Vererbungshierarchie für alle denkbaren Kombinationen wäre kombinatorisch nicht handhabbar.
Das Decorator Pattern ermöglicht es stattdessen, Boni und Mali zur Laufzeit dynamisch zu stapeln, ohne die \texttt{Unit}-Klasse zu verändern (vgl. \cite{entwurfsmuster}).

Die abstrakte Klasse \texttt{UnitDecorator} definiert die Methoden \texttt{getAttackBonus()}, \texttt{getDefenseBonus()} und \texttt{getMovementBonus()}, die jeweils die \texttt{Unit}-Instanz erhalten, sowie \texttt{tick()} für den Rundenablauf und \texttt{isExpired()} zur Lebenszeit-Überprüfung.
\texttt{Unit} hält eine \texttt{List\allowbreak<UnitDecorator>} und integriert die Boni transparent in seine Getter: \texttt{getAttack()} summiert \texttt{baseStats.attack()} mit dem \texttt{getAttackBonus()} aller nicht abgelaufenen Dekoratoren; analog für Verteidigung und Bewegung (Minimum~1).

Es gibt zwei Decorator-Arten: \texttt{SynergyBonus} ist permanent (isExpired() gibt stets \texttt{false} zurück) und berechnet den Bonus dynamisch anhand benachbarter Verbündeter -- beim Bewegen wird er entfernt und neu hinzugefügt.
Die sechs temporären Dekoratoren laufen nach zwei Runden ab: \texttt{AttackBuffDecorator} (+2~ATK), \texttt{AttackDebuffDecorator} ($-2$~ATK), \texttt{DefenseBuffDecorator} (+2~DEF), \texttt{DefenseDebuffDecorator} ($-2$~DEF), \texttt{HastenedDecorator} (+1~MOV) und \texttt{SlowedDecorator} ($-1$~MOV).
Der \texttt{BuffDebuffManager} wählt zu Beginn jedes Spielerzugs per Zufall eine lebende Einheit der aktiven Fraktion aus und wendet mit rundenweise steigender Wahrscheinlichkeit (Formel: $p = \min(0{,}60,\ 0{,}03 + (r-1) \cdot 0{,}06)$) einen zufälligen Effekt aus dem Buff- oder Debuff-Pool an.
Im UI werden aktive Effekte in geschweiften Klammern hinter dem Einheitennamen angezeigt.

\begin{figure}[H]
	\centering
	\includegraphics[width=\textwidth]{images/Decorator_Pattern___ElementarClash-Decorator_Pattern___ElementarClash_Unit_Buffs_Debuffs__GoF__4__Dynamically_Attaching_Responsibilities}
	\caption{Decorator Pattern für temporäre Buffs und Debuffs}
	\label{fig:decorator-pattern}
\end{figure}

\subsection{Verhaltensmuster}
\label{sec:verhaltensmuster}

\subsubsection{Strategy (Max Meier)}
\label{subsec:strategy}
% Verantwortlich: @mmukex
Einheiten unterschiedlicher Fraktionen besitzen verschiedene Bewegungs- und Angriffsregeln: Luft-Einheiten fliegen über Gelände hinweg, Feuer-Einheiten bewegen sich auf Lava günstiger, Fernkämpfer prüfen Sichtlinien.
Das Strategy Pattern kapselt diese Regeln in austauschbare Algorithmen hinter einem gemeinsamen Interface (vgl. \cite{gof}).

\texttt{MovementStrategy} definiert \texttt{calculateMovementCost(Terrain)} und \texttt{canMoveTo()}.
\texttt{GroundMovementStrategy} berücksichtigt fraktionsspezifische Geländemodifikatoren (z.\,B. Feuer auf Lava: Kosten~1, Wasser auf Lava: Kosten~3), \texttt{FlyingMovementStrategy} setzt alle Kosten pauschal auf~1.
\texttt{AttackStrategy} definiert \texttt{canAttack()}, \texttt{calculateBaseDamage()} und \texttt{getAttackRange()}.
\texttt{MeleeAttackStrategy} prüft die Manhattan-Distanz, \texttt{RangedAttackStrategy} validiert zusätzlich die Sichtlinie und blockiert Angriffe durch Wald-Gelände -- es sei denn, die Einheit besitzt \texttt{ignoresForestDefense} (vgl. \texttt{RangedAttackStrategy.java Z.\,52ff}).
Alle 12 konkreten Einheiten setzen ihre Strategien explizit im Konstruktor: 4~Einheiten erhalten \texttt{FlyingMovementStrategy} (alle Luft-Einheiten und Phönix), 5~Einheiten \texttt{RangedAttackStrategy} (Flammen-Bogenschütze, Frost-Magier, Sturm-Rufer, Himmels-Wächter, Terra-Schamane).
Die Getter in \texttt{Unit} liefern \texttt{GroundMovementStrategy} bzw. \texttt{MeleeAttackStrategy} als Fallback-Default (vgl. \texttt{Unit.java Z.\,142ff}).

\begin{figure}[H]
	\centering
	\includegraphics[width=\textwidth]{images/Strategy_Pattern___ElementarClash}
	\caption{Strategy Pattern für Bewegung und Angriff}
	\label{fig:strategy-pattern}
\end{figure}

\subsubsection{State (Christian Stiens)}
\label{subsec:state}
% Anwendung: Spielphasen (Setup, Event, InProgress, GameOver)
% Begründung: Zustandsabhängiges Verhalten
% UML-Diagramm
% Verantwortlich: @crstmkt
ElementarClash besitzt einen klar definierten Spielablauf aus vier Phasen, in denen jeweils unterschiedliche Aktionen erlaubt sind.
Ohne ein Zustandsmuster müsste die \texttt{Game}-Klasse bei jedem \texttt{executeCommand()}-Aufruf prüfen, welche Phase gerade aktiv ist -- was zu langen \texttt{if}-Kaskaden führt, die bei jeder neuen Phase erweitert werden müssen.
Das State Pattern verlagert das phasenspezifische Verhalten in separate Klassen und hält \texttt{Game} schlank (vgl. \cite{entwurfsmuster}).

Das Interface \texttt{GamePhaseState} deklariert \texttt{canExecuteCommand()}, \texttt{onEnter()}, \texttt{onExit()} sowie drei \texttt{transitionTo...()}-Methoden.
\texttt{Game.transitionToPhase(newPhase)} ruft \texttt{currentPhase.onExit(this)}, setzt die neue Phase und ruft \texttt{newPhase.onEnter(this)} auf -- der Phasenwechsel ist damit zuverlässig in einer einzigen Methode gekapselt.

Die vier konkreten Phasen sind:
\texttt{SetupPhase} (Singleton) blockiert alle Befehle und wird während der Spielaufbauphase im \texttt{GameBuilder} durchlaufen;
\texttt{PlayerTurnPhase} (Record mit \texttt{activeFaction}) aktiviert in \texttt{onEnter()} den \texttt{BuffDebuffManager} und setzt die Aktionszähler aller Einheiten der aktiven Fraktion zurück -- \texttt{canExecuteCommand()} delegiert die Validierung an \texttt{command.validate(game)};
\texttt{EventPhase} (Singleton) blockiert Befehle und feuert in \texttt{onEnter()} ein zufälliges Ereignis (Waldbrand, Geysir oder Erdbeben), bevor automatisch in die nächste \texttt{PlayerTurnPhase} gewechselt wird;
\texttt{GameOverPhase} (Record mit \texttt{winner}) ist der Terminalzustand -- alle Übergangsmethoden geben \texttt{this} zurück.

Zusätzlich modelliert das State Pattern die Einheitenzustände: \texttt{UnitState} mit den konkreten Zuständen \texttt{IdleState}, \texttt{MovingState}, \texttt{AttackingState}, \texttt{StunnedState} und \texttt{DeadState}.
Diese steuern, welche Aktionen eine Einheit innerhalb eines Spielerzugs noch ausführen darf.

\begin{figure}[H]
	\centering
	\includegraphics[width=\textwidth]{images/State_Pattern___ElementarClash-State_Pattern___ElementarClash_Game_Phases__GoF__6__Object_Behavior_Changes_Based_on_Internal_State}
	\caption{State Pattern für Spielphasen und Einheitenzustände}
	\label{fig:state-pattern}
\end{figure}

\subsubsection{Observer (Christian Stiens)}
\label{subsec:observer}
% Anwendung: Event-System (UI-Updates entkoppeln)
% Begründung: Lose Kopplung zwischen Spiellogik und Darstellung
% UML-Diagramm
% Verantwortlich: @crstmkt
Die Spiellogik in \texttt{Game} erzeugt Ereignisse -- Einheiten bewegen sich, greifen an, sterben, Gelände verändert sich -- die sofort in der Konsole dargestellt und gleichzeitig für eine spätere Protokollausgabe aufgezeichnet werden sollen.
Eine direkte Kopplung von \texttt{Game} an den \texttt{ConsoleGameRenderer} würde bedeuten, dass jede neue Ausgabeart (z.\,B. eine grafische Oberfläche oder ein Netzwerk-Log) Änderungen in der Spiellogik erfordert.
Das Observer Pattern löst dies durch eine lose Kopplung: \texttt{Game} kennt nur das Interface \texttt{GameObserver}, nicht seine konkreten Implementierungen (vgl. \cite{gof}).

\texttt{GameObserver} deklariert eine einzige Methode \texttt{onEvent(GameEvent)}.
\texttt{GameEvent} ist eine abstrakte Klasse mit \texttt{getEventType()} und \texttt{getDescription()}; acht konkrete Ereignisklassen decken alle spielrelevanten Übergänge ab: \texttt{GameStartedEvent}, \texttt{TurnStartedEvent}, \texttt{TurnEndedEvent}, \texttt{UnitMovedEvent}, \texttt{UnitAttackedEvent} (mit vollständigem \texttt{DamageResult}), \texttt{UnitDeathEvent}, \texttt{TerrainChangedEvent} und \texttt{GameOverEvent}.

\texttt{Game} hält eine \texttt{List\allowbreak<GameObserver>} und benachrichtigt alle registrierten Observer per \texttt{notifyObservers(event)} an klar definierten Punkten: in \texttt{startGame()}, \texttt{endTurn()}, \texttt{moveUnitInternal()}, \texttt{handleUnitDeath()} sowie im \texttt{AttackCommand} und im \texttt{Battlefield} bei Geländeänderungen.
Zwei konkrete Observer sind registriert: \texttt{ConsoleGameRenderer} verarbeitet jedes Ereignis in einer \texttt{switch}-Verzweigung und gibt es formatiert auf der Konsole aus; \texttt{EventLogObserver} sammelt alle Ereignisse in einer Liste, die am Spielende auf Wunsch vollständig ausgegeben werden kann.
Beide Observer werden im \texttt{GameController}-Konstruktor per \texttt{game.addObserver()} registriert, ohne dass \texttt{Game} dafür angepasst werden muss.

\begin{figure}[H]
	\centering
	\includegraphics[width=\textwidth]{images/Observer_Pattern___ElementarClash-Observer_Pattern___ElementarClash_Event_System__GoF__7__One_to_Many_Dependency_for_Event_Notification}
	\caption{Observer Pattern für UI-Updates und Event-System}
	\label{fig:observer-pattern}
\end{figure}

\subsubsection{Command (Max Meier)}
\label{subsec:command}
% Verantwortlich: @mmukex
Pro Runde stehen jeder Einheit zwei Aktionen zur Verfügung: Bewegen und Angreifen.
Der Spieler kann einzelne Aktionen rückgängig machen (Undo) und wiederholen (Redo).
Das Command Pattern kapselt Aktionen als Objekte mit \texttt{execute()}- und \texttt{undo()}-Methoden (vgl. \cite{gof}).

Das Interface \texttt{Command} deklariert \texttt{validate(Game)}, \texttt{execute(Game)} und \texttt{undo(Game)}.
\texttt{MoveCommand} speichert die vorherige Position und stellt sie bei \texttt{undo()} wieder her (vgl. \texttt{MoveCommand.java Z.\,77ff}).
\texttt{AttackCommand} sichert die Lebenspunkte des Ziels, nutzt den \texttt{DamageCalculator} (Chain of Responsibility, vgl. Abschnitt~\ref{subsec:chain}) und heilt das Ziel bei \texttt{undo()} auf den gespeicherten Wert zurück (vgl. \texttt{AttackCommand.java Z.\,86ff}).
Die \texttt{CommandHistory} verwaltet zwei Stacks (\texttt{executedCommands} für Undo, \texttt{undoneCommands} für Redo); \texttt{push()} leert den Redo-Stack, da ein neuer Befehl die Redo-Verzweigung verwirft (vgl. \texttt{CommandHistory.java Z.\,32ff}).
Der \texttt{CommandExecutor} validiert den Befehl, führt ihn aus und protokolliert ihn in der Historie (vgl. \texttt{CommandExecutor.java Z.\,22ff}).
Bei Rundenwechsel leert \texttt{clearHistory()} beide Stacks (Per-Turn-Rollback).

\begin{figure}[H]
	\centering
	\includegraphics[width=\textwidth]{images/Command_Pattern___ElementarClash}
	\caption{Command Pattern für Undo/Redo}
	\label{fig:command-pattern}
\end{figure}

\subsubsection{Chain of Responsibility (Christian Stiens)}
\label{subsec:chain}
% Anwendung: Schadensberechnung-Pipeline
% Begründung: Modulare Handler-Kette für Modifikatoren
% UML-Diagramm
% Verantwortlich: @crstmkt
Jeder Angriff in ElementarClash berücksichtigt mehrere unabhängige Schadensmodifikatoren: den strategiespezifischen Grundschaden, fraktionsbasierte Stärke- und Schwächerelationen, Geländeboni für Angreifer und Verteidiger sowie aktive Dekoratoren der angreifenden Einheit.
Diese Modifikatoren in einer einzelnen Methode zu berechnen würde eine enge Kopplung aller beteiligten Systeme erzeugen; jeder neue Modifikator (z.\,B. ein Wettereffekt) würde Änderungen an zentralem Code erfordern.
Das Chain of Responsibility Pattern kapselt jeden Modifikator in einem eigenen Handler und ermöglicht so eine erweiterbare, modulare Pipeline (vgl. \cite{entwurfsmuster}).

Die abstrakte Basisklasse \texttt{DamageHandler} hält eine Referenz auf den nächsten Handler und implementiert \texttt{handle()} so, dass Subklassen ihre Logik ausführen und dann \texttt{super.handle(context)} aufrufen, um die Kette fortzuführen.
Alle Handler lesen und schreiben auf einem gemeinsamen \texttt{DamageContext}-Objekt, das Angreifer, Verteidiger, Spielzustand und alle Zwischenergebnisse kapselt.
Die Kette wird einmalig im Konstruktor von \texttt{DamageCalculator} über fluente \texttt{setNext()}-Aufrufe aufgebaut und mit \texttt{handlerChain.handle(context)} gestartet.

Die fünf Handler in ihrer festen Reihenfolge:
\textbf{(1)} \texttt{BaseDamageHandler} liest den Grundschaden über das Strategy Pattern (\texttt{attacker.getAttackStrategy().calculateBaseDamage()}) aus.
\textbf{(2)} \texttt{FactionAdvantageHandler} schlägt Angreifer- und Verteidigerfraktion in einer statischen Korrelationsmatrix nach und multipliziert den Schaden mit $\times 1{,}25$ (Vorteil), $\times 0{,}75$ (Nachteil) oder $\times 1{,}0$ (neutral) (vgl. Tabelle 4).
\textbf{(3)} \texttt{TerrainEffectHandler} integriert das Visitor Pattern: Für Angreifer und Verteidiger wird je ein \texttt{TerrainVisitor} auf der aktuellen Zelle dispatcht und die Boni in den Kontext geschrieben.
\textbf{(4)} \texttt{SynergyBonusHandler} integriert das Decorator Pattern: er summiert \texttt{getAttackBonus()} aller nicht abgelaufenen Dekoratoren der angreifenden Einheit.
\textbf{(5)} \texttt{DefenseCalculationHandler} ist der terminale Handler und berechnet den Endschaden nach $\text{finalDamage} = \max(1,\ \text{totalAttack} - \text{totalDefense})$ -- ein Minimalschaden von~1 ist garantiert.
Das fertige Ergebnis wird als unveränderliches \texttt{DamageResult}-Record an den \texttt{AttackCommand} zurückgegeben.


\begin{figure}[H]
	\centering
	\includegraphics[scale=0.5]{images/Chain_of_Responsibility_Pattern___ElementarClash-Chain_of_Responsibility_Pattern___ElementarClash_Damage_Pipeline__GoF__9__Decoupled_Request_Processing_Through_Handler_Chain}
	\caption{Chain of Responsibility Pattern für modulare Schadensberechnung}
	\label{fig:chain-of-responsibility}
\end{figure}

\subsubsection{Visitor (Max Meier)}
\label{subsec:visitor}
% Verantwortlich: @mmukex
Jeder der fünf Geländetypen wirkt sich unterschiedlich auf die vier Fraktionen aus: Feuer-Einheiten erhalten auf Lava +2~Angriff, Wasser-Einheiten erleiden dort $-5$~LP/Runde, Erde und Luft bleiben unberührt.
20~Kombinationen (5~Gelände $\times$ 4~Fraktionen) direkt über \texttt{if}-Anweisungen zu modellieren wäre bei neuen Geländetypen fehleranfällig.
Das Visitor Pattern nutzt Double Dispatch, um beide Dimensionen (Gelände und Fraktion) ohne Fallunterscheidungen aufzulösen (vgl. \cite{gof}).

\texttt{TerrainVisitor} deklariert je eine Methode pro Fraktion: \texttt{visitFireUnit()}, \texttt{visitWaterUnit()}, \texttt{visitEarthUnit()}, \texttt{visitAirUnit()}.
\texttt{AbstractTerrainVisitor} liefert als Standard \texttt{TerrainEffectResult.NEUTRAL} und stellt Hilfsmethoden (\texttt{createAttackBonus()}, \texttt{createDefenseBonus()} etc.) bereit.
Fünf konkrete Visitor -- \texttt{LavaTerrainVisitor}, \texttt{IceTerrainVisitor}, \texttt{ForestTerrainVisitor}, \texttt{StoneTerrainVisitor}, \texttt{DesertTerrainVisitor} -- überschreiben nur die Methoden mit tatsächlichen Effekten.
\texttt{Unit.accept(TerrainVisitor)} realisiert den Double Dispatch per \texttt{switch} über die Fraktion (vgl. \texttt{Unit.java Z.\,168ff}).
Die \texttt{TerrainVisitorFactory} cached die fünf zustandslosen Visitor-Instanzen in einer \texttt{EnumMap} und liefert sie per \texttt{getVisitor(Terrain)} in $O(1)$ aus (vgl. \texttt{TerrainVisitorFactory.java Z.\,36ff}).
Das Ergebnis ist ein unveränderliches \texttt{TerrainEffectResult}-Record mit Angriffs-/Verteidigungsbonus, LP-pro-Runde-Effekt und optionaler Geländetransformation (z.\,B. Eis $\to$ Wüste bei Feuer-Einheiten).

\begin{figure}[H]
	\centering
	\includegraphics[width=\textwidth]{images/Visitor_Pattern___ElementarClash}
	\caption{Visitor Pattern für Geländeeffekte}
	\label{fig:visitor-pattern}
\end{figure}

\section{Implementierung}
\label{chap:implementierung}

% === ZIELUMFANG: ca. 4 Seiten ===

\subsection{Spielobjekte}
\label{sec:spielobjekte}

\subsubsection{Einheiten-Erzeugung (Christian Stiens)}
\label{subsec:einheiten-erzeugung}
% - Einheiten-Tabelle (alle 12 Einheiten mit Stats)
% - Pattern-Bezug: Factory Method
Jede der vier Fraktionen verfügt über drei Einheitentypen (vgl. Kapitel~2.3.1, Tabelle~\ref{tab:einheiten}).
Die Erzeugung erfolgt ausschließlich über die jeweilige \texttt{UnitFactory}-Subklasse (Factory Method Pattern), da alle konkreten Einheitenklassen package-private sind.

\begin{lstlisting}[language=Java, caption={UnitFactory -- Template Method und FireUnitFactory -- Factory Method}, label={lst:factory-method}]
// UnitFactory.java -- Abstract Creator
public abstract class UnitFactory {
    private final Faction faction;
    private int unitCounter = 0;

    public final Unit createUnit(UnitType type) {
        if (!isValidTypeForFaction(type)) {
            throw new IllegalArgumentException(
                "Unit type " + type + " not valid for " + faction);
        }
        String id = faction.name().substring(0, 1) + (++unitCounter);
        return createUnitInternal(id, type);   // Factory Method
    }

    protected abstract Unit createUnitInternal(String id, UnitType type);
    protected abstract boolean isValidTypeForFaction(UnitType type);
}

// FireUnitFactory.java -- Concrete Creator
public class FireUnitFactory extends UnitFactory {
    public FireUnitFactory() { super(Faction.FIRE); }

    @Override
    protected Unit createUnitInternal(String id, UnitType type) {
        return switch (type) {
            case INFERNO_WARRIOR -> new InfernoWarrior(id,
                    new UnitStats(100, 15, 5, 3, 1));
            case FLAME_ARCHER    -> new FlameArcher(id,
                    new UnitStats(70, 12, 3, 4, 3));
            case PHOENIX         -> new Phoenix(id,
                    new UnitStats(80, 10, 4, 5, 1));
            default -> throw new IllegalArgumentException(type.toString());
        };
    }
}
\end{lstlisting}

Listing~\ref{lst:factory-method} zeigt das zweistufige Muster.
\texttt{createUnit()} in \texttt{UnitFactory} ist die finale Template-Method: Sie validiert den \texttt{UnitType} per \texttt{isValidTypeForFaction()}, generiert eine eindeutige ID (z.\,B. \texttt{"F1"} für die erste Feuer-Einheit, \texttt{"W2"} für die zweite Wasser-Einheit) und delegiert die eigentliche Instanziierung an die abstrakte Factory Method \texttt{createUnitInternal()}.
\texttt{FireUnitFactory} implementiert diese per \texttt{switch}-Ausdruck und übergibt die fraktionsspezifischen Stat-Werte als \texttt{UnitStats}-Record direkt an den Konstruktor der konkreten, package-privaten Einheitenklasse.
Der \texttt{GameBuilder} hält ausschließlich Referenzen auf die abstrakte \texttt{UnitFactory} und \texttt{Unit}; konkrete Klassen sind für ihn nicht sichtbar.

\subsubsection{Spielfeld \& Gelände (Max Meier)}
\label{subsec:spielfeld-gelaende}
% - Pattern-Bezug: Builder, Composite, Visitor
Die Erzeugung des Spielfelds (Anforderung F1, vgl. Tabelle~\ref{tab:funktionale-anforderungen}) erfolgt über den \texttt{GameBuilder} (Builder Pattern).
Über eine Fluent-API werden zunächst Fraktionen registriert, optional eine benutzerdefinierte Geländeverteilung gesetzt und Einheiten hinzugefügt.
Die \texttt{build()}-Methode validiert die Konfiguration, erzeugt das \texttt{Battlefield} und platziert die Einheiten in den Ecken-Spawn-Zonen.

\begin{lstlisting}[language=Java, caption={GameBuilder -- Fluent-API und Build-Methode}, label={lst:builder-build}]
public GameBuilder withFactions(Faction... factions) {
    validateFactionCount(factions.length);
    this.factions.addAll(Arrays.asList(factions));
    return this;
}

public Game build() {
    validate();
    Battlefield battlefield = createAndInitializeBattlefield();
    Game game = new Game(battlefield);
    placeUnitsOnBattlefield(game);
    setInitialFaction(game);
    return game;
}

private void validate() {
    validateFactionsPresent();
    validateEachFactionHasUnits();
}
\end{lstlisting}

Listing~\ref{lst:builder-build} zeigt die Fluent-API und die \texttt{build()}-Methode.
\texttt{withFactions()} validiert die Fraktionsanzahl und gibt \texttt{this} zurück.
\texttt{build()} prüft, ob jede Fraktion Einheiten besitzt, initialisiert das \texttt{Battlefield} mit der Geländeverteilung und platziert die Einheiten in 3$\times$3-Spawn-Regionen an den Ecken (vgl. \texttt{GameBuilder.java Z.\,164ff}).

Das Spielfeld ist als Composite-Hierarchie aufgebaut: Das Interface \texttt{BattlefieldComponent} wird von \texttt{Cell} (Leaf), \texttt{Region} (Composite) und \texttt{Battlefield} (Root-Composite) implementiert.

\begin{lstlisting}[language=Java, caption={Composite-Hierarchie -- Cell, Region und Battlefield}, label={lst:composite-hierarchy}]
// BattlefieldComponent.java -- Component-Interface
public interface BattlefieldComponent {
    List<Cell> cells();
    default void applyEffect(Consumer<Cell> effect) {
        cells().forEach(effect);
    }
}

// Cell.java -- Leaf
public class Cell implements BattlefieldComponent {
    @Override
    public List<Cell> cells() { return List.of(this); }
}

// Region.java -- Composite
public record Region(List<Cell> cells)
        implements BattlefieldComponent {
    @Override
    public List<Cell> cells() { return new ArrayList<>(cells); }
}

// Battlefield.java -- Root-Composite
public class Battlefield implements BattlefieldComponent {
    private final List<Region> rows;
    @Override
    public List<Cell> cells() {
        return rows.stream()
            .flatMap(row -> row.cells().stream()).toList();
    }
}
\end{lstlisting}

Listing~\ref{lst:composite-hierarchy} zeigt die drei Ebenen der Composite-Hierarchie.
\texttt{Cell} gibt als Leaf eine Einelementliste zurück, \texttt{Region} delegiert an ihre Zellen, \texttt{Battlefield} flacht alle Zeilen ab.
\texttt{applyEffect(Consumer<Cell>)} wendet denselben Effekt -- etwa Wald zu Lava bei einem Waldbrand-Event -- auf eine einzelne Zelle, eine 3$\times$3-Region oder das gesamte Spielfeld an (vgl. \texttt{BattlefieldComponent.java Z.\,53ff}).

Die Geländeeffekte auf Einheiten werden über das Visitor Pattern abgebildet.
Jeder Geländetyp besitzt einen eigenen \texttt{TerrainVisitor}, der fraktionsspezifische Boni oder Mali berechnet.

\begin{lstlisting}[language=Java, caption={Visitor -- TerrainVisitor und LavaTerrainVisitor}, label={lst:visitor-lava}]
// TerrainVisitor.java
public interface TerrainVisitor {
    TerrainEffectResult visitFireUnit(Unit unit);
    TerrainEffectResult visitWaterUnit(Unit unit);
    TerrainEffectResult visitEarthUnit(Unit unit);
    TerrainEffectResult visitAirUnit(Unit unit);
}

// LavaTerrainVisitor.java
public class LavaTerrainVisitor
        extends AbstractTerrainVisitor {
    private static final int FIRE_ATTACK_BONUS = 2;
    private static final int WATER_HP_DRAIN = -5;

    @Override
    public TerrainEffectResult visitFireUnit(Unit unit) {
        return createAttackBonus(FIRE_ATTACK_BONUS,
            unit.getName() + ": +2 Angriff auf Lava");
    }

    @Override
    public TerrainEffectResult visitWaterUnit(Unit unit) {
        return createPerTurnEffect(WATER_HP_DRAIN,
            unit.getName() + ": -5 LP/Runde auf Lava");
    }
    // visitEarthUnit(), visitAirUnit() -> NEUTRAL (geerbt)
}
\end{lstlisting}

Listing~\ref{lst:visitor-lava} zeigt das \texttt{TerrainVisitor}-Interface und den \texttt{LavaTerrainVisitor}.
Feuer-Einheiten erhalten +2~Angriff, Wasser-Einheiten $-5$~LP/Runde; Erde und Luft erben \texttt{NEUTRAL} aus \texttt{AbstractTerrainVisitor}.
\texttt{Unit.accept(TerrainVisitor)} realisiert den Double Dispatch per \texttt{switch} über die Fraktion (vgl. \texttt{Unit.java Z.\,168ff}).
Die \texttt{TerrainVisitorFactory} hält die fünf zustandslosen Visitor-Instanzen in einer \texttt{EnumMap} und liefert sie in $O(1)$ aus (vgl. \texttt{TerrainVisitorFactory.java Z.\,36ff}).

\subsection{Spiellogik}
\label{sec:spiellogik}

\subsubsection{Rundenablauf \& Phasen (Christian Stiens)}
\label{subsec:rundenablauf}
% - Aktionen pro Runde: 1× Bewegen, 1× Angreifen
% - Zugwechsel zwischen Fraktionen
% - Pattern-Bezug: State / Decorator

Der Spielablauf ist als Zustandsautomat über das State Pattern realisiert (Anforderung F3, vgl. Tabelle~\ref{tab:funktionale-anforderungen}).
Alle Phasenwechsel laufen über \texttt{Game.transitionToPhase()}, die \texttt{onExit()} der alten und \texttt{onEnter()} der neuen Phase aufruft -- der Mechanismus ist damit an genau einer Stelle konzentriert.

\begin{lstlisting}[language=Java, caption={State Pattern -- transitionToPhase() und PlayerTurnPhase}, label={lst:state-transition}]
// Game.java -- Context
private void transitionToPhase(GamePhaseState newPhase) {
    currentPhase.onExit(this);
    this.currentPhase = newPhase;
    newPhase.onEnter(this);
}

// PlayerTurnPhase.java -- Concrete State (Record)
public record PlayerTurnPhase(Faction activeFaction)
        implements GamePhaseState {

    @Override
    public void onEnter(Game game) {
        // Decorator Pattern: (De-)Buff mit steigender Wahrscheinlichkeit
        new BuffDebuffManager()
            .tryApplyRandomEffect(game, game.getRoundNumber());
        // Aktionszaehler aller Einheiten der Fraktion zuruecksetzen
        game.getUnitsOfFaction(activeFaction).forEach(Unit::resetTurn);
    }

    @Override
    public boolean canExecuteCommand(Game game, Command command) {
        return command.validate(game).isValid();
    }

    @Override
    public GamePhaseState transitionToEventPhase(Game game) {
        return EventPhase.getInstance();
    }

    @Override
    public GamePhaseState transitionToGameOver(Game game, Faction winner) {
        return new GameOverPhase(winner);
    }
}
\end{lstlisting}

Listing~\ref{lst:state-transition} zeigt den Phasenwechsel-Mechanismus und die zentrale \texttt{PlayerTurnPhase}.
Da \texttt{PlayerTurnPhase} ein Java-Record ist, enthält sie die aktive Fraktion unveränderlich als Komponente -- ein Wechsel der Fraktion erzeugt eine neue Instanz (\texttt{transitionToPlayerTurn()} gibt \texttt{new PlayerTurnPhase(faction)} zurück).
In \texttt{onEnter()} werden zwei Aufgaben erledigt: der \texttt{BuffDebuffManager} versucht, einen Dekorator auf eine Einheit anzuwenden (Decorator Pattern), und anschließend werden alle Einheiten der aktiven Fraktion für den neuen Zug zurückgesetzt -- konkret setzt \texttt{Unit.resetTurn()} den Aktionszähler auf~0 und wechselt den \texttt{UnitState} per State Pattern nach \texttt{IdleState}.
\texttt{canExecuteCommand()} delegiert vollständig an \texttt{command.validate(game)}, sodass die Phasenklasse keine Kenntnis der einzelnen Command-Typen benötigt.
Die \texttt{EventPhase} blockiert alle Befehle, feuert in \texttt{onEnter()} ein zufälliges Geländeereignis (Waldbrand, Geysir oder Erdbeben über \texttt{applyEffect()} des Composite-Spielfelds) und geht danach automatisch in die nächste \texttt{PlayerTurnPhase} über.



\subsubsection{Bewegung \& Angriffsstrategie (Max Meier)}
\label{subsec:bewegung-angriffsstrategie}
% - Pattern-Bezug: Strategy
Die Bewegungs- und Angriffslogik (Anforderung F4, vgl. Tabelle~\ref{tab:funktionale-anforderungen}) wird über das Strategy Pattern in zwei Hierarchien gekapselt.
Das Interface \texttt{MovementStrategy} definiert die Berechnung von Geländekosten und die Zugvalidierung.
\texttt{GroundMovementStrategy} berücksichtigt fraktionsspezifische Modifikatoren (z.\,B. Feuer auf Lava: Kosten~1 statt~2, Wasser auf Lava: Kosten~3), während \texttt{FlyingMovementStrategy} alle Geländekosten pauschal auf~1 setzt.

\begin{lstlisting}[language=Java, caption={Strategy Pattern -- MovementStrategy und FlyingMovementStrategy}, label={lst:strategy-movement}]
// MovementStrategy.java
public interface MovementStrategy {
    double calculateMovementCost(Terrain terrain);
    boolean canMoveTo(Game game, Position current,
                      Position target, int maxMovement);
}

// FlyingMovementStrategy.java
public class FlyingMovementStrategy
        implements MovementStrategy {
    private static final double FLYING_TERRAIN_COST = 1.0;

    @Override
    public double calculateMovementCost(Terrain terrain) {
        return FLYING_TERRAIN_COST;
    }

    @Override
    public boolean canMoveTo(Game game, Position current,
                             Position target, int maxMov) {
        if (game.isPositionOccupied(target)) return false;
        return current.manhattanDistanceTo(target) <= maxMov;
    }
}
\end{lstlisting}

Listing~\ref{lst:strategy-movement} zeigt das \texttt{MovementStrategy}-Interface und die \texttt{FlyingMovementStrategy}.
Fliegende Einheiten (alle Luft-Einheiten sowie der Phönix der Feuer-Fraktion) erhalten diese Strategie und ignorieren damit sämtliche Geländekosten -- ihre effektive Reichweite entspricht stets dem Basis-Bewegungswert.

Listing~\ref{lst:strategy-ground} zeigt die \texttt{GroundMovementStrategy} mit fraktionsspezifischen Geländemodifikatoren (vgl. \texttt{GroundMovementStrategy.java Z.\,44ff}).

\begin{lstlisting}[language=Java, caption={GroundMovementStrategy -- fraktionsspezifische Geländekosten}, label={lst:strategy-ground}]
public class GroundMovementStrategy
        implements MovementStrategy {

    private static final double FIRE_LAVA_COST = 1.0;
    private static final double FIRE_ICE_COST = 2.0;
    private static final double WATER_ICE_COST = 1.0;
    private static final double WATER_LAVA_COST = 3.0;

    private final Faction faction;

    @Override
    public double calculateMovementCost(Terrain terrain) {
        double baseCost = terrain.getMovementCost();
        return switch (faction) {
            case FIRE  -> applyFireModifiers(terrain, baseCost);
            case WATER -> applyWaterModifiers(terrain, baseCost);
            case EARTH -> applyEarthModifiers(terrain, baseCost);
            default    -> baseCost;
        };
    }

    private double applyFireModifiers(Terrain t, double base) {
        return switch (t) {
            case LAVA -> FIRE_LAVA_COST;   // Kosten 1 statt 2
            case ICE  -> FIRE_ICE_COST;    // Kosten 2 statt 3
            default   -> base;
        };
    }

    private double applyWaterModifiers(Terrain t, double base) {
        return switch (t) {
            case ICE  -> WATER_ICE_COST;   // Kosten 1 statt 3
            case LAVA -> WATER_LAVA_COST;  // Kosten 3 statt 2
            default   -> base;
        };
    }

    private double applyEarthModifiers(Terrain t, double base) {
        return switch (t) {
            case STONE -> 2.0;             // Kosten 2 statt 3
            default    -> base;
        };
    }
}
\end{lstlisting}

Die Kosten variieren je nach Fraktion und Gelände: Feuer auf Lava~1 statt~2, Wasser auf Eis~1 statt~3, dafür auf Lava~3 statt~2, Erde auf Stein~2 statt~3.
Fliegende Einheiten zahlen pauschal~1.

\texttt{AttackStrategy} definiert die Methoden \texttt{canAttack()}, \texttt{calculateBaseDamage()} und \texttt{getAttackRange()}.
\texttt{MeleeAttackStrategy} prüft die Manhattan-Distanz gegen die Reichweite der Einheit (typischerweise~1).
\texttt{RangedAttackStrategy} validiert zusätzlich die Sichtlinie: Wald-Gelände auf dem Pfad zwischen Angreifer und Ziel blockiert den Angriff, sofern die Einheit nicht über die Eigenschaft \texttt{ignoresForestDefense} verfügt -- wie etwa der Flammen-Bogenschütze (vgl. \texttt{RangedAttackStrategy.java Z.\,52ff}).
Alle 12 konkreten Einheiten setzen ihre Strategien explizit im Konstruktor: 4~Einheiten erhalten \texttt{FlyingMovementStrategy} (Phoenix, WindDancer, StormCaller, SkyGuardian), 5~Einheiten \texttt{RangedAttackStrategy} -- davon der Flammen-Bogenschütze mit \texttt{ignoresForestDefense=true}, die übrigen mit \texttt{false}.
Die Getter in \texttt{Unit} (Z.\,142ff) enthalten \texttt{GroundMovementStrategy} bzw. \texttt{MeleeAttackStrategy} als Fallback-Default.

\subsubsection{Schadensberechnung (Christian Stiens)}
\label{subsec:schadensberechnung}
% - Schadensformel
% - Gelände-Modifikatoren
% - Elementare Stärken/Schwächen (+25%/-25%)
% - Pattern-Bezug: Chain of Responsibility

Die Schadensberechnung wird über das Chain of Responsibility Pattern abgebildet.
Jeder Angriff durchläuft dabei eine fest definierte Kette von fünf Handlern, die gemeinsam auf dem \texttt{DamageContext}-Objekt arbeiten und dieses schrittweise anreichern.

\begin{lstlisting}[language=Java, caption={DamageCalculator -- Aufbau und Start der Handler-Kette}, label={lst:damage-calculator}]
// DamageCalculator.java
public class DamageCalculator {
    private final DamageHandler handlerChain;

    public DamageCalculator() {
        // Kette in fester Reihenfolge aufbauen (Reihenfolge entscheidend)
        this.handlerChain = new BaseDamageHandler();
        handlerChain
            .setNext(new FactionAdvantageHandler())
            .setNext(new TerrainEffectHandler())     // Visitor Pattern
            .setNext(new SynergyBonusHandler())      // Decorator Pattern
            .setNext(new DefenseCalculationHandler());
    }

    public DamageResult calculateDamage(Unit attacker,
                                        Unit target, Game game) {
        DamageContext context =
            new DamageContext(attacker, target, game);
        handlerChain.handle(context);   // Kette starten
        return context.toResult();      // unveraendl. Record
    }
}
\end{lstlisting}

Listing~\ref{lst:damage-calculator} zeigt, wie \texttt{DamageCalculator} die Kette einmalig im Konstruktor über fluente \texttt{setNext()}-Aufrufe aufbaut.
\texttt{calculateDamage()} erstellt einen frischen \texttt{DamageContext} für jeden Angriff, startet die Kette mit \texttt{handlerChain.handle(context)} und wandelt das fertige Kontextobjekt in ein unveränderliches \texttt{DamageResult}-Record um.

Die abstrakte Basisklasse \texttt{DamageHandler} hält eine Referenz auf den jeweils nächsten Handler und leitet den Kontext nach getaner Arbeit via \texttt{super.handle()} weiter (vgl. \texttt{DamageHandler.java Z.\,22, 40ff}).
Der erste Handler, \texttt{BaseDamageHandler}, liest den Grundschaden über das Strategy Pattern aus (\texttt{attacker.getAttackStrategy().calculateBaseDamage()}) und schreibt ihn in den Kontext (vgl. \texttt{BaseDamageHandler.java Z.\,14ff}).
Der \texttt{FactionAdvantageHandler} wendet anschließend einen Multiplikator aus einer statischen Fraktionsmatrix an, etwa $\times 1{,}25$ bei Vorteil oder $\times 0{,}75$ bei Nachteil (vgl. \texttt{FactionAdvantageHandler.java Z.\,25ff}, Tabelle~\ref{tab:fraktions-korrelationsmatrix}).

Der \texttt{TerrainEffectHandler} integriert das Visitor Pattern: Für Angreifer und Verteidiger wird je ein \texttt{TerrainVisitor} dispatcht, dessen Boni in den Kontext einfließen (vgl. \texttt{TerrainEffectHandler.java Z. 21ff}).
Der \texttt{SynergyBonusHandler} greift auf die aktiven Dekoratoren der angreifenden Einheit zu und summiert deren Angriffsboni, womit das Decorator Pattern in die Kette eingebunden wird (vgl. \texttt{SynergyBonusHandler.java Z. 16ff}).
Als terminaler Handler berechnet der \texttt{DefenseCalculationHandler} schließlich den Endschaden nach der Formel $\text{finalDamage} = \max(1,\ \text{totalAttack} - \text{totalDefense})$ und schreibt das Ergebnis in den Kontext (vgl. \texttt{DefenseCalculationHandler.java Z. 15ff}).
Der fertige Kontext wird anschließend in ein unveränderliches \texttt{DamageResult}-Record umgewandelt und vom \texttt{AttackCommand} genutzt, um den Schaden auf die Zieleinheit anzuwenden und die Observer zu benachrichtigen (vgl. \texttt{DamageContext.java Z. 90ff}, \texttt{AttackCommand.java Z. 91ff}).

\subsubsection{Buffs/Debuffs (Christian Stiens)}
\label{subsec:buffs-debuffs}
% - Temporäre (De-)Buffs zur Laufzeit
% - Pattern-Bezug: Decorator

Im Verlauf des Spiels wird mit rundenweise steigender Wahrscheinlichkeit (max. 60\,\%) zu Beginn jedes Spielerzugs einer von sechs Effekten auf eine zufällig ausgewählte, lebende Einheit der aktiven Fraktion angewendet (Decorator Pattern, vgl. \texttt{BuffDebuffManager.java Z.\,41ff}).

\begin{lstlisting}[language=Java, caption={Decorator Pattern -- Unit.getAttack() und AttackBuffDecorator}, label={lst:decorator-unit}]
// Unit.java -- Dekoratoren transparent integriert
public int getAttack() {
    int attack = baseStats.attack();
    for (UnitDecorator decorator : decorators) {
        if (!decorator.isExpired()) {
            attack += decorator.getAttackBonus(this);
        }
    }
    return attack;
}

public void removeExpiredDecorators() {
    decorators.removeIf(UnitDecorator::isExpired);
}

// AttackBuffDecorator.java -- Konkrete Decorator-Implementierung
public class AttackBuffDecorator extends UnitDecorator {
    private static final int BONUS = 2;
    private static final int DURATION = 2;
    private int remainingRounds = DURATION;

    @Override public int getAttackBonus(Unit unit) { return BONUS; }
    @Override public int getDefenseBonus(Unit unit) { return 0; }
    @Override public int getMovementBonus(Unit unit) { return 0; }
    @Override public boolean isExpired() { return remainingRounds <= 0; }
    @Override public void tick() { remainingRounds--; }
    @Override public String getDescription() {
        return "+2 Attack (" + remainingRounds + " rounds left)";
    }
}
\end{lstlisting}

Listing~\ref{lst:decorator-unit} zeigt die Integration des Decorator Patterns in \texttt{Unit}.
\texttt{getAttack()} iteriert über alle Dekoratoren, filtert abgelaufene per \texttt{isExpired()} und akkumuliert deren \texttt{getAttackBonus(this)} auf den Basiswert aus \texttt{baseStats}; \texttt{getDefense()} und \texttt{getMovement()} folgen demselben Muster.
\texttt{AttackBuffDecorator} gewährt +2~ATK für zwei Runden: \texttt{tick()} dekrementiert \texttt{remainingRounds} am Ende jedes Spielerzugs, \texttt{isExpired()} signalisiert bei \texttt{remainingRounds\,$\leq$\,0} das Ende der Wirkdauer.
\texttt{removeExpiredDecorators()} entfernt alle abgelaufenen Einträge in einem einzigen \texttt{removeIf()}-Aufruf (vgl. \texttt{Unit.java Z.\,210ff}).
Die aktiven Effekte sind im UI in geschweiften Klammern hinter dem Einheitennamen sichtbar, z.\,B. \texttt{\{+2 Attack (1 rounds left)\}}.

\subsubsection{Undo/Redo (Max Meier)}
\label{subsec:undo-redo}
% - Pattern-Bezug: Command
Das Undo/Redo-System (Anforderung F4, vgl. Tabelle~\ref{tab:funktionale-anforderungen}) basiert auf dem Command Pattern.
Jede Spieleraktion wird als \texttt{Command}-Objekt mit eigener Validierungs-, Ausführungs- und Rücknahmelogik gekapselt.
Der \texttt{CommandExecutor} validiert ein Command, führt es aus und übergibt es an die \texttt{CommandHistory}.

\begin{lstlisting}[language=Java, caption={CommandHistory -- Undo/Redo mit zwei Stacks}, label={lst:command-history}]
public class CommandHistory {
    private final Deque<Command> executedCommands;
    private final Deque<Command> undoneCommands;

    public void push(Command command) {
        executedCommands.push(command);
        undoneCommands.clear(); // Neuer Befehl verwirft Redo
    }

    public Command popForUndo() {
        if (executedCommands.isEmpty()) return null;
        Command cmd = executedCommands.pop();
        undoneCommands.push(cmd);
        return cmd;
    }

    public Command popForRedo() {
        if (undoneCommands.isEmpty()) return null;
        Command cmd = undoneCommands.pop();
        executedCommands.push(cmd);
        return cmd;
    }

    public void clear() {
        executedCommands.clear();
        undoneCommands.clear();
    }
}
\end{lstlisting}

Listing~\ref{lst:command-history} zeigt die \texttt{CommandHistory} mit ihren zwei Stacks.
Der \texttt{executedCommands}-Stack enthält alle ausgeführten Befehle der aktuellen Runde; bei einem Undo wird der oberste Befehl auf den \texttt{undoneCommands}-Stack verschoben und seine \texttt{undo()}-Methode aufgerufen.
Ein Redo entnimmt umgekehrt vom \texttt{undoneCommands}-Stack und führt den Befehl erneut aus.
Sobald ein neuer Befehl per \texttt{push()} hinzugefügt wird, wird der Redo-Stack geleert, da der neue Befehl eine Verzweigung der bisherigen Aktionsfolge darstellt.
Bei Rundenwechsel leert \texttt{clear()} beide Stacks, sodass der Spieler nur Aktionen der laufenden Runde rückgängig machen kann (Per-Turn-Rollback).

\texttt{MoveCommand} speichert im Feld \texttt{previousPosition} die Ausgangsposition der Einheit und setzt diese bei \texttt{undo()} zurück (vgl. \texttt{MoveCommand.java Z.\,77ff}).
\texttt{AttackCommand} speichert die Lebenspunkte des Ziels vor dem Angriff (\texttt{targetPreviousHealth}) und heilt das Ziel bei \texttt{undo()} um die Differenz (vgl. \texttt{AttackCommand.java Z.\,86ff}).
Beide Commands dekrementieren bei Undo den Aktionszähler der Einheit, sodass die Aktion erneut zur Verfügung steht.

\begin{lstlisting}[language=Java, caption={AttackCommand -- execute() und undo()}, label={lst:attack-command}]
public class AttackCommand implements Command {

    private final Unit actor;
    private final Unit target;

    private int targetPreviousHealth;
    private int damageDealt;
    private boolean wasExecuted;

    @Override
    public void execute(Game game) {
        this.targetPreviousHealth = target.getCurrentHealth();

        DamageCalculator calculator = new DamageCalculator();
        DamageResult result =
            calculator.calculateDamage(actor, target, game);

        this.damageDealt = result.totalDamage();
        target.takeDamage(result.totalDamage());
        actor.incrementActionsThisTurn();
        actor.startAttacking();

        game.notifyObservers(
            new UnitAttackedEvent(actor, target, result));

        if (!target.isAlive()) {
            game.handleUnitDeath(target);
        }
        this.wasExecuted = true;
    }

    @Override
    public void undo(Game game) {
        if (!wasExecuted) {
            throw new IllegalStateException(
                "Cannot undo command that wasn't executed");
        }
        int healthToRestore =
            targetPreviousHealth - target.getCurrentHealth();
        if (healthToRestore > 0) {
            target.heal(healthToRestore);
        }
        actor.decrementActionsThisTurn();
    }
}
\end{lstlisting}

Listing~\ref{lst:attack-command} zeigt den \texttt{AttackCommand}.
\texttt{execute()} sichert \texttt{targetPreviousHealth}, berechnet den Schaden über den \texttt{DamageCalculator} (Chain of Responsibility) und benachrichtigt die Observer per \texttt{UnitAttackedEvent}.
\texttt{undo()} heilt das Ziel um die Differenz und dekrementiert den Aktionszähler.

\subsection{Benutzeroberfläche (Christian Stiens)}
\label{sec:ui}

% - ASCII-Darstellung des Spielfelds
% - Eingabe-Aktionen: [B]ewegen, [A]ngreifen, [U]ndo, [R]edo, [Z]ug beenden
% - Ggf. Screenshot
% - Pattern-Bezug: Observer
Das 10$\times$10-Spielfeld wird als ASCII-Raster in der Konsole gerendert (Anforderung F6, vgl. Tabelle~\ref{tab:funktionale-anforderungen}).
Eine grafische Oberfläche hätte den zeitlichen Rahmen des Projektes gesprengt; die Konsolenausgabe ist vollständig funktional und zeigt alle spielrelevanten Informationen.

\begin{lstlisting}[language=Java, caption={Observer Pattern -- ConsoleGameRenderer.onEvent() und GameController}, label={lst:observer-ui}]
// ConsoleGameRenderer.java -- Concrete Observer
public class ConsoleGameRenderer
        implements GameRenderer, GameObserver {

    @Override
    public void onEvent(GameEvent event) {
        switch (event.getEventType()) {
            case UNIT_MOVED    -> handleUnitMoved(
                                    (UnitMovedEvent) event);
            case UNIT_ATTACKED -> handleUnitAttacked(
                                    (UnitAttackedEvent) event);
            case UNIT_DIED     -> handleUnitDeath(
                                    (UnitDeathEvent) event);
            case TERRAIN_CHANGED -> handleTerrainChanged(
                                    (TerrainChangedEvent) event);
            case TURN_STARTED,
                 TURN_ENDED    -> { /* Separator-Ausgabe */ }
            case GAME_OVER     -> handleGameOver(
                                    (GameOverEvent) event);
            default            -> { }
        }
    }

    private void handleUnitAttacked(UnitAttackedEvent event) {
        DamageResult r = event.getDamageResult();
        System.out.println("[Attack] " + event.getDescription());
        System.out.println("  Base: " + r.baseDamage()
            + ", Faction: x" + r.factionMultiplier()
            + ", Terrain: +"  + r.terrainAttackBonus()
            + ", Synergy: +"  + r.synergyBonus()
            + ", Defense: -"  + r.totalDefense()
            + " = "           + r.totalDamage() + " total");
    }
}

// GameController.java -- Observer-Registrierung
public GameController(Game game) {
    this.renderer = new ConsoleGameRenderer();
    this.eventLog = new EventLogObserver();
    game.addObserver((ConsoleGameRenderer) renderer);
    game.addObserver(eventLog);            // 2. Observer
}
\end{lstlisting}

Listing~\ref{lst:observer-ui} zeigt die Observer-Kopplung zwischen Spiellogik und Darstellung.
\texttt{ConsoleGameRenderer} implementiert gleichzeitig \texttt{GameRenderer} (für den direkten \texttt{render(game)}-Aufruf nach jeder Eingabe) und \texttt{GameObserver} (für ereignisgetriebene Ausgaben während der Ausführung).
In \texttt{onEvent()} werden alle acht \texttt{EventType}-Werte per \texttt{switch} auf private Handler-Methoden verteilt; \texttt{handleUnitAttacked()} gibt dabei die vollständige Schadensaufschlüsselung aus dem \texttt{DamageResult}-Record aus, das die Chain of Responsibility befüllt hat.
Im \texttt{GameController}-Konstruktor werden beide Observer per \texttt{game.addObserver()} registriert: \texttt{ConsoleGameRenderer} für die Echtzeit-Konsolenausgabe und \texttt{EventLogObserver} zum Aufzeichnen aller Ereignisse.
Der \texttt{GameController} liest Spielereingaben über \texttt{ConsoleUI} aus und verteilt sie per \texttt{switch} auf die entsprechenden Command-Objekte (\texttt{MoveCommand}, \texttt{AttackCommand}), die über \texttt{game.executeCommand()} ausgeführt und in der \texttt{CommandHistory} protokolliert werden (vgl. \texttt{GameController.java Z.\,63ff}).

\begin{figure}[H]
    \centering
    \includegraphics[width=\textwidth]{images/UI}
    \caption{Beispielhafte Konsolenausgabe eines Spielfeldes}
    \label{fig:ui}
\end{figure}




\section{Zusammenfassung (Christian Stiens)}
\label{chap:zusammenfassung}

% === ZIELUMFANG: ca. 1 Seite ===
\subsection{Fazit}
\label{sec:fazit}
% Fazit:
% - 10 GoF-Patterns erfolgreich eingesetzt
% - Modulare, erweiterbare Architektur erreicht
% - Spielbare Version mit Undo/Redo

Das Ziel, ein lauffähiges Programm (Spiel) unter Berücksichtigung von 10 selbstgewählten Design Pattern nach GoF zu erstellen wurde erreicht.
Durch die erfolgreiche Implementierung der Design Pattern ist eine modulare, erweiterbare Architektur geschaffen worden.
Die eingereichte Version ist im Sinne der geplanten Implementierung vollständig und lauffähig.

% Ausblick (kurz):
% - Neue Fraktionen/Einheiten ohne Core-Änderungen möglich
% - Potentielle Erweiterungen: KI-Gegner, grafische UI
\subsection{Ausblick}
\label{sec:ausblick}
Wie bereits erwähnt sind Erweiterungen ohne Core Änderungen möglich.
Erweiterungen können zum Beispiel weitere Einheiten je Fraktion oder zusätzliche (De-)Buffs sein, welche zu Beginn eines Spielzugs angewendet werden (vgl. Kapitel 3.2.2. Decorator Pattern).
Weitere denkbare Erweiterungen könnten zum Beispiel KI Gegner oder eine echte grafische UI sein.

% ============================================
% Literaturverzeichnis
% ============================================
\begin{thebibliography}{9}

\bibitem{gof} 
Gamma, E., Helm, R., Johnson, R., Vlissides, J.:
\textit{Design Patterns: Elements of Reusable Object-Oriented Software}.
Addison-Wesley, 1994.

\bibitem{entwurfsmuster}
Matthias Geirhos:
\textit{Entwurfsmuster Das umfassende Handbuch}
Rheinwerk Computing, 1. Auflage 2015, 4., korrigierter Nachdruck 2021, ISBN 978-3-8362-2762-9

% Weitere Quellen nach Bedarf

\end{thebibliography}


\end{document}
