\chapter{Zusammenfassung (Christian Stiens)}
\label{chap:zusammenfassung}

% === ZIELUMFANG: ca. 1 Seite ===
\section{Fazit}
\label{sec:fazit}
% Fazit:
% - 10 GoF-Patterns erfolgreich eingesetzt
% - Modulare, erweiterbare Architektur erreicht
% - Spielbare Version mit Undo/Redo

Das Ziel, ein lauffähiges Programm (Spiel) unter Berücksichtigung von 10 selbstgewählten Design Pattern nach GoF wurde erreicht.
Durch die erfolgreiche Implementierung der Design Pattern ist eine modulare, erweiterbare (z.B. weitere Einheiten je Fraktion) Architektur geschaffen worden.
Die eingereichte Version ist im Sinne der geplanten Implementierung vollständig und lauffähig.

% Ausblick (kurz):
% - Neue Fraktionen/Einheiten ohne Core-Änderungen möglich
% - Potentielle Erweiterungen: KI-Gegner, grafische UI
\section{Ausblick}
\label{sec:ausblick}
Wie bereits erwähnt sind Erweiterungen ohne Core Änderungen möglich.
Erweiterungen können zum Beispiel weitere Einheiten je Fraktion oder zusätzliche (De-)Buffs sein, welche zu Beginn eines Spielzugs angewendet werden (vgl. Kapitel 3.2.2. Decorator Pattern).
Weitere denkbare Erweiterungen könnten zum Beispiel KI Gegner oder eine echte Grafische UI sein.

% ============================================
% Literaturverzeichnis
% ============================================
\begin{thebibliography}{9}

\bibitem{gof} 
Gamma, E., Helm, R., Johnson, R., Vlissides, J.:
\textit{Design Patterns: Elements of Reusable Object-Oriented Software}.
Addison-Wesley, 1994.

% Weitere Quellen nach Bedarf

\end{thebibliography}
